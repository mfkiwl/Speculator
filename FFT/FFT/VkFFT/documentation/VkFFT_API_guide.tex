%% LyX 2.3.4.3 created this file.  For more info, see http://www.lyx.org/.
%% Do not edit unless you really know what you are doing.
\documentclass[12pt,english]{article}
\usepackage{amsmath}
\usepackage{amssymb}
\usepackage{fontspec}
\usepackage{geometry}
\geometry{verbose,tmargin=2.5cm,bmargin=2.5cm,lmargin=2.5cm,rmargin=2.5cm}
\usepackage{stackrel}
\usepackage[unicode=true]
 {hyperref}

\makeatletter
%%%%%%%%%%%%%%%%%%%%%%%%%%%%%% User specified LaTeX commands.
\usepackage{sourcecodepro}
\usepackage[parfill]{parskip}
\usepackage{enumitem}
\setlist[itemize]{leftmargin=*}
\usepackage{minted}
\usepackage{mdframed}
\definecolor{bg}{rgb}{0.95,0.95,0.95}
\usepackage{hyperref}
\hypersetup{
    colorlinks,
    citecolor=black,
    filecolor=black,
    linkcolor=black,
    urlcolor=black
}

\makeatother

\usepackage{polyglossia}
\setdefaultlanguage[variant=american]{english}
\begin{document}
\UseRawInputEncoding
\begin{titlepage} 	
\centering 	
\vspace{1cm} 	
{\scshape\LARGE VkFFT - Vulkan/CUDA/HIP/OpenCL/Level Zero Fast Fourier Transform library \par} 		
\vspace{1.5cm} 	
{\huge\bfseries API guide with examples\par} 	
\vspace{2cm} 	
{\Large Dmitrii Tolmachev\par} 	
	
\vspace{1cm} 	
{\large April 2021, version 1.2.24\par} 
\end{titlepage}

\newpage{}

\tableofcontents{}

\newpage{}

\section{Introduction}

This document describes VkFFT - Vulkan/CUDA/HIP/OpenCL/Level Zero
Fast Fourier Transform library. It describes the features and current
limitations of VkFFT, explains the API and compares it to other FFT
libraries (like FFTW and cuFFT) on the set of examples. It is by no
means the final version, so if there is something unclear - feel free
to contact me (dtolm96@gmail.com), so I can update it. 

\newpage{}

\section{Using the VkFFT API}

This chapter will cover the basics of VkFFT. Fourier transform of
a sequence is called Discrete Fourier Transform (DFT). It is defined
by the following formula:
\begin{equation}
X_{k}=\stackrel[n=0]{N-1}{\sum}x_{n}e^{-\frac{2\pi i}{N}nk}=\mathrm{DFT}{}_{N}(x_{n},k),\label{eq:dft}
\end{equation}

where $x_{n}$is the input sequence, $N$ is the length of the input
sequence and $k\in[0,N-1],k\in\mathbb{Z}$is the output index, corresponding
to frequency in Fourier space. Corresponding to that, inverse DFT
is defined as following:
\begin{equation}
x_{n}=\stackrel[k=0]{\mathrm{N-1}}{\sum}X_{k}e^{\frac{2\pi i}{N}nk}=\mathrm{iDFT}{}_{N}(X_{k},n)
\end{equation}

VkFFT follows the same definitions as FFTW and cuFFT - forward FFT
has the exponent sign $-1$, while the inverse has the exponent sign
$1$. Note, that inverse transform by default is unnormalized, so
to get the input sequence after FFT + iFFT, the user has to divide
the result by $N$.

\subsection{Installing VkFFT}

VkFFT is distributed as a header-only library. The installation process
consists of the following steps:
\begin{enumerate}
\item \begin{flushleft}
Copy vkFFT.h file into one of the directories included in the user's
project.
\par\end{flushleft}
\item Define VKFFT\_BACKEND as a number corresponding to the API used in
the user's project: 0 - Vulkan, 1 - CUDA, 2 - HIP, 3 - OpenCL, 4 -
Level Zero. Definition is done like:\begin{mdframed}[backgroundcolor=bg]
\begin{minted}[tabsize=4,obeytabs,breaklines]{make}
-DVKFFT_BACKEND=X
\end{minted}
\end{mdframed} in GCC or as \begin{mdframed}[backgroundcolor=bg]
\begin{minted}[tabsize=4,obeytabs,breaklines]{cmake}
set(VKFFT_BACKEND 1 CACHE STRING "0 - Vulkan, 1 - CUDA, 2 - HIP, 3 - OpenCL, 4 - Level Zero")
\end{minted}
\end{mdframed}in CMake.
\item Depending on the API backend, the project must use additional libraries
for run-time compilation:
\begin{enumerate}
\item Vulkan API: SPIRV, glslang and Vulkan. Define VK\_API\_VERSION to
the available Vulkan version. Sample CMakeLists can look like this:\begin{mdframed}[backgroundcolor=bg]
\begin{minted}[tabsize=4,obeytabs,breaklines]{cmake}
find_package(Vulkan REQUIRED)
target_compile_definitions(${PROJECT_NAME} PUBLIC -DVK_API_VERSION=11)#10 - Vulkan 1.0, 11 - Vulkan 1.1, 12 - Vulkan 1.2 
target_include_directories(${PROJECT_NAME} PUBLIC ${CMAKE_CURRENT_SOURCE_DIR}/glslang-master/glslang/Include/) 
add_subdirectory(${CMAKE_CURRENT_SOURCE_DIR}/glslang-master)

target_include_directories(${PROJECT_NAME} PUBLIC ${CMAKE_CURRENT_SOURCE_DIR}/vkFFT/)
add_library(VkFFT INTERFACE)
target_compile_definitions(VkFFT INTERFACE -DVKFFT_BACKEND=0)

target_link_libraries(${PROJECT_NAME} PUBLIC SPIRV glslang Vulkan::Vulkan VkFFT)
\end{minted}
\end{mdframed}
\item CUDA API: CUDA and NVRTC. Sample CMakeLists can look like this:\begin{mdframed}[backgroundcolor=bg]
\begin{minted}[tabsize=4,obeytabs,breaklines]{cmake}
find_package(CUDA 9.0 REQUIRED) 	
enable_language(CUDA) 	
set_property(TARGET ${PROJECT_NAME} PROPERTY CUDA_ARCHITECTURES 35 60 70 75 80 86) 	
target_compile_options(${PROJECT_NAME} PUBLIC "$<$<COMPILE_LANGUAGE:CUDA>:SHELL: 	
	-DVKFFT_BACKEND=${VKFFT_BACKEND} 	
	-gencode arch=compute_35,code=compute_35
	-gencode arch=compute_60,code=compute_60
	-gencode arch=compute_70,code=compute_70
	-gencode arch=compute_75,code=compute_75
	-gencode arch=compute_80,code=compute_80
	-gencode arch=compute_86,code=compute_86>")
set_target_properties(${PROJECT_NAME} PROPERTIES CUDA_SEPARABLE_COMPILATION ON)
set_target_properties(${PROJECT_NAME} PROPERTIES CUDA_RESOLVE_DEVICE_SYMBOLS ON)

find_library(CUDA_NVRTC_LIB libnvrtc nvrtc HINTS "${CUDA_TOOLKIT_ROOT_DIR}/lib64" "${LIBNVRTC_LIBRARY_DIR}" "${CUDA_TOOLKIT_ROOT_DIR}/lib/x64" /usr/lib64 /usr/local/cuda/lib64)

target_include_directories(${PROJECT_NAME} PUBLIC ${CMAKE_CURRENT_SOURCE_DIR}/vkFFT/)
add_library(VkFFT INTERFACE)
target_compile_definitions(VkFFT INTERFACE -DVKFFT_BACKEND=1)

target_link_libraries(${PROJECT_NAME} PUBLIC ${CUDA_LIBRARIES} cuda ${CUDA_NVRTC_LIB} VkFFT)
\end{minted}
\end{mdframed}
\item HIP API: HIP and HIPRTC. Sample CMakeLists can look like this:\begin{mdframed}[backgroundcolor=bg]
\begin{minted}[tabsize=4,obeytabs,breaklines]{cmake}
list(APPEND CMAKE_PREFIX_PATH /opt/rocm/hip /opt/rocm)
find_package(hip)

target_include_directories(${PROJECT_NAME} PUBLIC ${CMAKE_CURRENT_SOURCE_DIR}/vkFFT/)
add_library(VkFFT INTERFACE)
target_compile_definitions(VkFFT INTERFACE -DVKFFT_BACKEND=2)
#target_compile_definitions(${PROJECT_NAME} PUBLIC -DVKFFT_OLD_ROCM) #ROCm versions before 4.5 needed kernel include of hiprtc

target_link_libraries(${PROJECT_NAME} PUBLIC hip::host VkFFT)
\end{minted}
\end{mdframed}
\item OpenCL API: OpenCL. Sample CMakeLists can look like this:\begin{mdframed}[backgroundcolor=bg]
\begin{minted}[tabsize=4,obeytabs,breaklines]{cmake}
find_package(OpenCL REQUIRED)

target_include_directories(${PROJECT_NAME} PUBLIC ${CMAKE_CURRENT_SOURCE_DIR}/vkFFT/)
add_library(VkFFT INTERFACE)
target_compile_definitions(VkFFT INTERFACE -DVKFFT_BACKEND=3)

target_link_libraries(${PROJECT_NAME} PUBLIC OpenCL::OpenCL VkFFT)
\end{minted}
\end{mdframed}
\item Level Zero API: Level Zero; Clang and llvm-spirv must be in the system
path (for kernel compilation). Sample CMakeLists can look like this:\begin{mdframed}[backgroundcolor=bg]
\begin{minted}[tabsize=4,obeytabs,breaklines]{cmake}
set(LevelZero_LIBRARY "/usr/lib/x86_64-linux-gnu/")
set(LevelZero_INCLUDE_DIR "/usr/include/")
find_library(
	LevelZero_LIB
	NAMES "ze_loader"
	PATHS ${LevelZero_LIBRARY}
	PATH_SUFFIXES "lib" "lib64"
	NO_DEFAULT_PATH
  )
find_path(
	LevelZero_INCLUDES
	NAMES "ze_api.h"
	PATHS ${LevelZero_INCLUDE_DIR}
	PATH_SUFFIXES "include" 
	NO_DEFAULT_PATH
  )
target_include_directories(${PROJECT_NAME} PUBLIC ${LevelZero_INCLUDES})
add_library(VkFFT INTERFACE)
target_compile_definitions(VkFFT INTERFACE -DVKFFT_BACKEND=4)

target_link_libraries(${PROJECT_NAME} PUBLIC LevelZero VkFFT)
\end{minted}
\end{mdframed}
\end{enumerate}
\end{enumerate}

\subsection{Fourier Transform Setup}

VkFFT follows a plan structure like FFTW/cuFFT with a notable difference
- there is a unified interface to all transforms. This means that
there are no separate functions like fftPlan1D/fftPlan2D/fftPlanMany/etc.
The initialization is done through a single configuration struct -
VkFFTConfiguration. Each parameter of it will be covered in detail
in this document. Plans in VkFFT are called VkFFTApplication and they
are created with a unified initializeVkFFT call. 

As the code is written in C, don't forget to zero-initialize used
structs!

During the initializeVkFFT(VkFFTApplication{*} app, VkFFTConfiguration
inputLaunchConfiguration) call VkFFT performs kernel generation and
compilation from scratch (kernel reuse may be added later). The overall
process of initialization looks like this:
\begin{enumerate}
\item Get device parameters, perform default initialization of internal
copy of configuration struct inside the VkFFTApplication, then fill
in user-defined parameters from inputLaunchConfiguration.~VkFFTApplication
is passed as a pointer, so initializeVkFFT modifies the user-provided
application.
\item By default, there are two internal FFT plans created - inverse and
forward. Multidimensional FFT is done as a combination of 1D FFTs
in each axis direction. For each axis, the VkFFTPlanAxis function
is called.
\item VkFFTPlanAxis configures parameters for each axis. It may perform
additional memory allocations (see: memory allocated by VkFFT).
\item shaderGenVkFFT generates corresponding to the axis code in a char
buffer (each axis may require more than one kernel: see Four-step
FFT, Bluestein's algorithm for FFT). 
\item Code is then compiled with the run-time compiler of the specified
backend.
\end{enumerate}
Once the plan is no longer need, a call to the deleteVkFFT function
frees all the allocated resources. There are no processes launched
that continue to work outside of the VkFFT related function calls.

\subsection{Fourier Transform types and their definitions}

VkFFT supports commonly used Complex to complex (C2C), real to complex
(R2C), complex to real (C2R) transformations and real to real (R2R)
Discrete Cosine Transformations of types II, III and IV. VkFFT uses
the same definitions as FFTW, except for the multidimensional FFT
axis ordering: in FFTW dimensions are ordered with the decrease in
consecutive elements stride, while VkFFT does the opposite - the first
axis is the non-strided axis (the one that has elements located consecutively
in memory with no gaps, usually named as the X-axis). So, in FFTW
dimensions are specified as ZYX and in VkFFT as XYZ. This felt more
logical to me - no matter if there are 1, 2 or 3 dimensions, the user
can always find the axis with the same stride at the same position.
This choice doesn't require any modification in the user's data management
- just provide the FFT dimensions in the reverse order to VkFFT. 

In addition to up to the 3 dimensions of FFT, VkFFT supports two forms
of batching: the number of coordinates and the number of systems.
The choice of two distinct batching ways is made to support matrix-vector
convolutions, where the kernel is presented as a matrix. Overall,
the layout of VkFFT can be described as WHDCN - width, height, depth,
coordinate and number of systems (in order of increasing strides,
starting with 1 for width). Coordinate and number of systems can be
1, if the user has 1 as one of the FFT dimensions, the user can omit
it from setup altogether as FFT of size 1 produces the same number
as the input. Often, the coordinate part of the layout is not used,
so the main batching is done by specifying N.

VkFFT assumes that complex numbers are stored consecutively in memory:
RIRIRI... where R denotes the real part of the complex number and
I denotes the imaginary part. There is no difference between using
a float2/double2/half2 container or access memory as float/double/half
as long as the byte order remains the same.

This section and the next one will cover the basics of VkFFT data
layouts and memory management. 

\subsubsection{C2C transforms}

The base FFT algorithm - C2C in VkFFT has the same definition as FFTW.
Forward FFT has the exponent sign $-1$, while the inverse has the
exponent sign $1$. By default, the inverse transform is unnormalized.
$N_{x}N_{y}N_{z}$ complex numbers map to $N_{x}N_{y}N_{z}$ complex
numbers and no additional padding is required. The resulting data
order will be the same as in FFTW/cuFFT, unless special parameters
are provided in configuration (see: advanced memory management)

\subsubsection{R2C/C2R transforms}

R2C/C2R transforms can be explained as C2C transforms with imaginary
part set to zero. They exploit Hermitian symmetry of the result: $X_{k}=X_{N-k}^{*}$
on the non-strided axis (the one that has elements located consecutively
in memory with no gaps). This results in s reduction of required memory
to store the complex result - we may only store $floor(\frac{N_{x}}{2})+1$
complex numbers instead of $N_{x}$. However, this results in memory
requirements mismatch between input and output in R2C: $floor(\frac{N_{x}}{2})+1$
complex elements will require $N_{x}+2$ real numbers worth of memory
for even $N_{x}$ and $N_{x}+1$ real numbers worth of memory for
odd $N_{x}$. For C2R the situation is reversed. There are two approaches
to this problem: pad each sequence of the non-strided axis with zeros
to the required length or use out-of-place mode. More information
on how to do this will be given in the next section.

\subsubsection{R2R (DCT) transforms}

R2R transforms in VkFFT are implemented in the form of Discrete cosine
transforms of types I, II, III and IV. Their definitions and transforms
results match FFTW:
\begin{enumerate}
\item DCT-I: $X_{k}=x_{0}+(-1)^{k}x_{N-1}+2\stackrel[n=1]{N-2}{\sum}x_{n}cos(\frac{\pi}{N-1}nk)$,
inverse of DCT-I (itself)
\item DCT-II: $X_{k}=2\stackrel[n=1]{N-1}{\sum}x_{n}cos(\frac{\pi}{N}(n+\frac{1}{2})k)$,
inverse of DCT-III
\item DCT-III: $X_{k}=x_{0}+2\stackrel[n=1]{N-1}{\sum}x_{n}cos(\frac{\pi}{N}n(k+\frac{1}{2}))$,
inverse of DCT-II
\item DCT-IV: $X_{k}=2\stackrel[n=0]{N-1}{\sum}x_{n}cos(\frac{\pi}{N}(n+\frac{1}{2})(k+\frac{1}{2}))$,
inverse of DCT-IV (itself)
\end{enumerate}
R2R transforms are performed by redefinition of them to the C2C transforms
(internal C2C sequence length can be different from the input R2R
sequence length). R2R transform performs a one-to-one mapping between
real numbers, so they don't require stride management, unlike R2C/C2R.

\subsection{Memory management, data layouts for different transforms}

\subsubsection{VkFFT buffers}

VkFFT allows for explicit control over the data flow, which makes
both in-place and out-of-place transforms possible. Buffers are passed
to VkFFT as VkBuffer pointer in Vulkan, as double void pointers in
CUDA/HIP/Level Zero and as cl\_mem pointer in OpenCL. This is done
to maintain a uniform data pattern because some of the buffers can
be allocated automatically. 

The main buffer is called buffer and it always has to be provided,
either during the plan creation or when the plan is executed. All
calculations are performed in this buffer and it is always overwritten.
To do calculations out-of-place, VkFFT provides an option to specify
inputBuffer/outputBuffer buffer. The logic behind their usage is fairly
simple - the user specifies inputBuffer if the input data has to be
read from a buffer, different from the main buffer. As the data is
read only once and nothing is written back to the inputBuffer, this
allows doing truly out-of-place transformations. The same logic applies
to outputBuffer with the difference that it is responsible for the
absolute last write of the VkFFT. It is possible to use all three
buffers to create complex data management paths.

It must be noted, that sometimes FFT can not be done inside one buffer
(see: Four-Step FFT algorithm, Bluestein's algorithm). To compute
FFT in these cases, there exists tempBuffer buffer and data is transferred
between the main buffer and tempBuffer during the FFT execution. The
ordering of transfers between the main buffer and tempBuffer is done
in such a way, so the initial data read and final data write are obeying
the configuration from the previous paragraph. Users can allocate
tempBuffer themselves of some memory that does not have any useful
information at the time of FFT execution (the tempBuffer size can
depend on the configuration, so this is a rather advanced operation
- read more in the advanced memory management section) or allow VkFFT
to manage tempBuffer allocation itself (tempBuffer will be freed at
the deleteVkFFT call).

To compute convolutions and cross-correlations, a kernel buffer has
to be specified. It must have the same layout as the result of the
FFT transform.

\subsubsection{VkFFT buffers strides. A special case of R2C/C2R transforms}

To have better control of memory, the user can specify the strides
between consecutive elements of different axis for H (height), D (depth)
and C (coordinate) parts of the WHDCN layout (W (width) stride is
fixed to be 1, N (number of systems) stride will be consecutive of
C in memory if C is used, otherwise N will propagate the previous
non-uniform stride multiplied by the corresponding axis length). Strides
are specified not in bytes, but in the element type used - similar
to the way how the user would access the corresponding element in
the array. If all elements are consecutive in C2C, stride for H will
be equal to the FFT length of W axis, stride for D will be multiplication
of first two FFT axis lengths, stride for C will be multiplication
of first three FFT axis lengths, etc. These are the default values
of C2C and R2R strides if they are not explicitly specified.

One of the main use-cases of strides comes to solve the R2C/C2R Hermitian
symmetry H stride mismatch - for real space, it is equal to $N_{x}$
real elements and for the frequency space it is equal to $floor(\frac{N_{x}}{2})+1$
complex numbers. So, with strides it is possible to use a buffer,
padded to $2\cdot(floor(\frac{N_{x}}{2})+1)$ real elements in H stride
(all elements between $N_{x}$ and $2\cdot(floor(\frac{N_{x}}{2})+1)$
will not be read so it does not matter what data is there before the
write stage). All other strides are done as a multiplication between
the previous stride and the number of elements in the previous axis.
These are the default values of R2C/C2R strides if they are not explicitly
specified.

It is possible to specify separate sets of strides for all user-defined
buffers: bufferStride for the main buffer, inputStride for input buffer,
outputStride for output buffer (kernel stride is assumed to be the
same as bufferStride, tempBuffer strides are configured automatically). 

For an out-of-place R2C FFT, there is no need to pad buffer with real
numbers, but user must specify H stride there (as it differs to default
one) - $N_{x}$ real elements for real space and $floor(\frac{N_{x}}{2})+1$
complex numbers for the frequency space.

An out-of-place C2R FFT is a more tricky transform. In the multidimensional
case, the main buffer will be written to and read from multiple times.
The intermediate stores have a complex layout, which requires more
space than the output real layout, so in order not to modify the input
data, there exist two options. First, pad the real data layout has
to $2\cdot(floor(\frac{N_{x}}{2})+1)$ real elements in H stride (complex
buffer will be used as inputBuffer, real buffer as buffer). Second,
use the third buffer, so both input and output buffers have their
original layouts (complex buffer will be used as inputBuffer, the
main buffer for calculations is buffer and output real buffer as outputBuffer).
If you use inverseReturnToInputBuffer option, where R2C is configured
to read from input buffer and C2R is configured to write to the input
buffer; C2R will modify the buffer it reads from in some cases (see
issue \href{https://github.com/DTolm/VkFFT/issues/58\#issuecomment-1007205682}{\#58})

\subsection{VkFFT algorithms}

VkFFT implements a wide range of algorithms to compute different types
of FFTs but all of them can be reduced to a mixed-radix Cooley-Tukey
FFT algorithm in the Stockham autosort form. The main idea behind
it is to decompose the sequence as a set of primes, for each of which
FFT can be written down exactly. As of now, VkFFT has radix implementations
for primes up to 13, so all C2C sequences decomposable into a multiplication
of such primes will be done purely with the Stockham algorithm. Below
additional algorithms and their use-cases are described.

\subsubsection{Bluestein's algorithm}

A complex algorithm that is used in cases where the sequence is not
decomposable with implemented radix butterflies (currently - primes
up to 13). It is derived by replacing $nk=\left(n^{2}+k^{2}-(n-k)^{2}\right)/2$
in \ref{eq:dft}:

\begin{align}
X_{k} & =\left(e^{-\pi i\frac{k^{2}}{N}}\right)\stackrel[n=0]{N-1}{\sum}\left(x_{n}e^{-\pi i\frac{n^{2}}{N}}\right)\left(e^{\pi i\frac{\left(k-n\right)^{2}}{N}}\right)=b_{k}^{\ast}\stackrel[n=0]{N-1}{\sum}a_{n}b_{k-n}\label{eq:dft-1}\\
a_{n} & =x_{n}b_{n}^{\ast}\\
b_{n} & =e^{\pi i\frac{n^{2}}{N}}
\end{align}

Here FFT is represented as a convolution between two sequences: $a_{n}$
and $b_{n}$, which can be performed by the means of convolution theorem:

\begin{equation}
F\{a\ast b\}=F\{a\}\cdot F\{b\}
\end{equation}

By padding $a_{n}$ and $b_{n}$ to a sequence length decomposable
with implemented radix butterflies with a size of at least $2N-1$
(because the length of $b_{n}$ is $2N-1$), we can perform FFT of
any length. FFT of $b_{n}$ can be precomputed, so overall this algorithm
requires at least 4x the computations and more memory transfers. This
algorithm can be combined with all other algorithms implemented in
VkFFT. If an FFT can not be done in a single upload, a tempBuffer
has to be allocated (because the logical FFT buffer size is bigger
than the original system).

\subsubsection{The Four-Step FFT algorithm}

GPUs and CPUs have a hierarchical memory model - the closer memory
to the unit that performs the computations, the faster its speed and
the lower the size. So it is advantageous to split FFTs, not to the
lowest primes, but to some bigger multiplication of those primes,
then upload this subsequence to the closest cache level to the cores
and do the final prime split there. The absolute lowest level is the
register file, however, it does not allow for thread communications
outside the warp. For this purpose, modern GPUs employ shared memory
- a fast memory with a bank structure that is visible to all threads
in a thread block. The usual sizes of it change on a scale from 16KB
to 192KB and it is often beneficial to use it fully. However, if the
full sequence can not fit inside the shared memory, FFT has to be
done in multiple uploads - with the Four Step FFT algorithm. The main
idea behind it is to represent a big 1D sequence as a 2D (or 3D for
the three-upload scheme) FFT - we first do FFT along the columns,
then the rows, then transpose the result and multiply by a special
set of phase vectors. Similar decomposition idea as the main Cooley-Tukey
algorithm. However, performing transpositions in-place is a complicated
task - especially for a non-trivial ratio between dimensions. It will
also require an additional read/write stage, as it can not be merged
with the last write of the FFT algorithm. The easiest and the most
performant solution is to use a tempBuffer (it is the main reason
for having this functionality, actually) and store intermediate FFT
results out-of-place. This way the last transposition step can be
merged with the write step, as we can overwrite the output buffer
without losing data.

To estimate if your sequence size is single upload or not, divide
the amount of available shared memory (48KB - Nvidia GPUs with Vulkan/OpenCL
API, 64KB - AMD GPUs, 100KB - Nvidia GPUs in CUDA API) by the complex
size used for calculations (8 byte - single precision, 16 byte - double
precision). For 64KB of shared memory, we get 8192 as max single upload
single-precision non-strided FFT, 4096 for double precision. For strided
axes (H and D parts of the layout) these numbers have to be divided
by 4 and 2 respectively to achieve coalescing, resulting in 2048 length
for single upload in both precisions. For more information on coalescing
see: coalescing API reference.

In the case of the Four-Step FFT algorithm, tempBuffer size has to
be at least the same as the default main buffer size. It does not
matter how many uploads are in the Four Step FFT algorithm - only
a single tempBuffer is required. In this document, all systems that
can fit in the shared memory entirely and be done without the Four
Step FFT algorithm (and multiple uploads) are called single upload
systems.

If the last transposition is not required (the output data is allowed
to be in not unshuffled form) it can be disabled during the configuration
phase. This way tempBuffer will not be needed and all computations
will be done in-place (unless Bluestein's algorithm is used). An example
use-case of this is convolutions - if the kernel is computed with
the same operation ordering, point-wise multiplication in the frequency
domain is not dependent on the correct data ordering and the inverse
FFT will restore the original layout.

\subsubsection{R2C/C2R FFTs}

A typical approach to a single upload R2C/C2R system is to just set
the imaginary part to zero inside the shared memory and do a simple
C2C transform. This doesn't affect the amount of memory transferred
from VRAM and is not a bad approach as FFT is a memory-bound algorithm,
however, this can be improved in multidimensional (in HDCN part of
the layout) case by the composition of a single C2C sequence from
two real sequences and some write for R2C/read for C2R post-processing.
Both of these algorithms are implemented in VkFFT. Note, that R2C/C2R
only affects the non-strided axis (W). All strided axes are still
done as C2C.

\subsubsection{R2C/C2R multi-upload FFT algorithm}

For even sequences there exists an easy mapping between R2C/C2R FFTs
and the C2C of half the size. In this case, all even indices (starting
from 0) are read as the real values of a complex number and all odd
indices are read as the imaginary values. This C2C sequence can be
done with the help of the Four-Step FFT algorithm. When FFT is done,
separate post-processing for R2C/pre-processing for C2R is applied.

\subsubsection{R2R Discrete Cosine Transforms}

There exist many different mappings between DCT and FFT. As of now,
VkFFT has the following algorithms implemented (all single-upload
for now):
\begin{itemize}
\item DCT-I - mapping between R2R and C2C of the $2N-2$ length. For non-strided
axis can use an optimization similar to the R2C/C2R multidimensional
case (setting the imaginary part to the next FFT sequence).
\item DCT-II/DCT-III - mapping between R2R and C2C of the same length. For
non-strided axis can use an optimization similar to the R2C/C2R multidimensional
case (setting the imaginary part to the next FFT sequence).
\item DCT-IV - for even sizes, mapping between R2R and C2C sequence of half-length.
For odd sizes mapping to the FFT of the same length (for non-strided
axis can use an optimization similar to the R2C/C2R multidimensional
case (setting the imaginary part to the next FFT sequence)).
\end{itemize}

\subsubsection{Register overutilization}

Not an FFT algorithm by itself, but an optimization to do bigger sequences
in a single upload instead of switching to the Four Step FFT algorithm.
The main idea behind it is to use a register file (which is often
bigger than the amount of shared memory) to store the sequence and
use shared memory only as a communication buffer. This is useful in
Vulkan and OpenCL APIs on Nvidia GPU, as they are only allowed to
allocate 48KB of shared memory with a register file having the size
of 256KB.

\subsubsection{Zero padding}

Not an FFT algorithm by itself, but a memory management optimization.
If the user's system has parts that are known to be zero - for example,
when an open system is modeled, to avoid a circular part of the FFT
system has to be padded with zeros up to 2x in each direction. VkFFT
can omit sequences full of zeros and don't perform the corresponding
memory transfers and computations, as the output result will be zero.
This way it is possible to get up to two times speed increase in the
2D case and up to 3x increase in the 3D case. 

\subsubsection{Convolution and cross-correlation support}

With the help of the Convolution theorem, which states that the Fourier
transform of a convolution is the pointwise product of signals Fourier
transforms, it is possible to perform convolution with $NlogN$ complexity,
compared to $N^{2}$ complexity of the simple multiplication approach.
This is extremely useful for kernels spanning more than 50 elements
in size. VkFFT can merge the last step FFT, kernel multiplication
in the Fourier domain and the first step of inverse FFT to provide
substantial memory transfer savings. Moreover, FFTs of big sequences
can be performed without data reordering, which results in a better
locality. 

\subsection{VkFFT accuracy}

To measure how VkFFT (single/double/half precision) results compare
to cuFFT/rocFFT (single/double/half precision) and FFTW (double precision),
multiple sets of systems covering full supported C2C/R2C+C2R/R2R FFT
range are filled with random complex data on the scale of {[}-1,1{]}
and one transform was performed on each system. Samples 11(single),
12(double), 13(half), 14(non-power of 2 C2C, single), 15(R2C+C2R,
single), 16(DCT-I/II/III/IV, single), 17(DCT-I/II/III/IV, double),
18(non-power of 2 C2C, double) are available in VkFFT Benchmark Suite
to perform VkFFT verification on any of the target platforms. Overall,
the Cooley-Tukey algorithm (Stockham autosort) exhibits logarithmic
relative error scaling, similar to those of other GPU FFT libraries.
Typically, the more computationally expensive algorithm is - the worse
its precision is. So, Bluestein's algorithm has lower accuracy than
Stockham autosort algorithm.

Single precision in VkFFT supports two modes of calculation - by using
the on-chip Special Function Units that can compute sines and cosines
on the go or by using the precomputed on CPU look-up tables. For Nvidia
and AMD GPUs, SFU provide great precision, while Intel iGPUs and mobile
GPUs must use LUT to perform FFTs correctly.

Double precision in VkFFT also supports two modes of calculation -
by using polynomial sincos approximation and computing them on-chip
or by using precomputed LUT as well. The second option is the better
one, as polynomial sincos approximation is too compute-heavy for modern
GPUs. It is selected by default on all devices.

Half precision is currently only supported in the Vulkan backend and
is often experiencing precision problems with the first number of
the resulting FFT sequence, which is the sum of all input numbers.
Half precision is implemented only as a memory trick - all on-chip
computations are done in single precision, but this doesn't help with
the first number problem. Half precision can use SFU or LUT as well.

VkFFT also supports mixed-precision operations, where memory storing
is done at lower precision, compared to the on-chip calculations.
For example, it is possible to read data in single precision, do calculations
in double and store data back in single precision. 

\subsection{VkFFT additional memory allocations}

In this section, all GPU memory allocations that are done by VkFFT
are described. There are up to three situations when VkFFT allocates
memory. All of the VkFFT allocated memory is freed at the deleteVkFFT
call.

\subsubsection{LUT allocations}

This memory is used to store precomputed twiddle factors and phase
vectors used during the computation. This buffer can have:
\begin{itemize}
\item twiddle factors for each radix stage of Stockham FFT calculation
\item phase vectors used in the Four Step FFT algorithm between stages
\item phase vectors used in DCT-II/III/IV to perform a mapping between R2R
and C2C
\end{itemize}
VkFFT manages LUT allocations by itself and they are performed during
the initializeVkFFT call. LUT are allocated per axis, though some
of them can be reused if the axes have the same LUT. Inverse and forward
FFT plans share the same LUT (conjugation is performed on-chip).

\subsubsection{The Four-Step FFT algorithm - tempBuffer allocation}

To perform the merging of the transposition with the last upload of
an axis, VkFFT requires additional memory to mimic an out-of-place
execution. This memory is located in tempBuffer and has to be of at
least the same size as the main buffer. It is possible for the users
to allocate it themselves, though if this is not done, VkFFT can do
the allocation automatically (the size of the tempBuffer will be the
same as the main buffer, unless the logical dimensions of FFT are
bigger than user-defined - then, it will allocate the system with
the minimal size, that can cover maximal logical system size used
in any of the axes - see next subsection).

\subsubsection{Bluestein's buffers allocation}

To do Bluestein's FFT algorithm, precomputed sequences $b_{n}=e^{\pi i\frac{n^{2}}{N}}$,
$FFT(b_{n})$ and $iFFT(b_{n})$ are required. For each axis, they
can be different and are computed separately (unless VkFFT can determine
that they match, then the buffers are allocated only once). Notably,
as Bluestein's algorithm pads the sequence length to at least $2N-1$,
if it can not be done in a single upload and the Four Step algorithm
has to be used, the intermediate storage required will be bigger than
the main buffer size. In this case, tempBuffer must always be allocated.
As the padded sequence can be different for each of the dimensions,
the required size of the tempBuffer will also vary. VkFFT determines
the biggest size needed among axes and allocated tempBuffer of this
size.

\newpage{}

\section{VkFFT API Reference}

This section covers error codes, API functions that can be used by
the user and configuration parameters.

\subsection{Return value VkFFTResult}

All VkFFT Library return values except for VKFFT\_SUCCESS are used
in case of a failure and provide information on what has gone wrong.
VkFFTResult is unified among different backends, though some of its
values may not be used in specific backends. Possible return values
of VkFFTResult are defined as following:

\begin{mdframed}[backgroundcolor=bg]
\begin{minted}[tabsize=4,obeytabs,breaklines]{C}
typedef enum VkFFTResult { 	
VKFFT_SUCCESS = 0,	// The VkFFT operation was successful
VKFFT_ERROR_MALLOC_FAILED = 1,	// Some malloc call inside VkFFT has failed. Report this to the GitHub repo
VKFFT_ERROR_INSUFFICIENT_CODE_BUFFER = 2,	// Generated kernel is bigger than default kernel array. Increase it with maxCodeLength parameter of configuration.
VKFFT_ERROR_INSUFFICIENT_TEMP_BUFFER = 3,	// Temporary string used in kernel generation is bigger than default temporary string array. Increase it with maxTempLength parameter of configuration.
VKFFT_ERROR_PLAN_NOT_INITIALIZED = 4,	// Code attempts to use uninitialized plan (it is zero inside VkFFTApplication)		
VKFFT_ERROR_NULL_TEMP_PASSED = 5,	// Internal kernel generation error
VKFFT_ERROR_INVALID_PHYSICAL_DEVICE = 1001,	// No physical device is provided (Vulkan API)
VKFFT_ERROR_INVALID_DEVICE = 1002,	// No device is provided (All APIs)
VKFFT_ERROR_INVALID_QUEUE = 1003,	// No queue is provided (Vulkan API)
VKFFT_ERROR_INVALID_COMMAND_POOL = 1004,	// No command pool is provided (Vulkan API)
VKFFT_ERROR_INVALID_FENCE = 1005,	// No fence is provided (Vulkan API)
VKFFT_ERROR_ONLY_FORWARD_FFT_INITIALIZED = 1006,	// VkFFT tries to access inverse FFT plan, when appliction is created with makeForwardPlanOnly flag
VKFFT_ERROR_ONLY_INVERSE_FFT_INITIALIZED = 1007,	// VkFFT tries to access forward FFT plan, when appliction is created with makeInversePlanOnly flag
VKFFT_ERROR_INVALID_CONTEXT = 1008,	// No context is provided (OpenCL API)
VKFFT_ERROR_INVALID_PLATFORM = 1009,	// No platform is provided (OpenCL API)
VKFFT_ERROR_EMPTY_FILE = 1011,
VKFFT_ERROR_EMPTY_FFTdim = 2001,	// Number of dimensions is not provided in the configuration
VKFFT_ERROR_EMPTY_size = 2002,	// Array of dimensions is not provided in the configuration
VKFFT_ERROR_EMPTY_bufferSize = 2003,	// Buffer size has to be provided during the application creation
VKFFT_ERROR_EMPTY_buffer = 2004,	// Buffer has te be specified either at the application creation stage or during launch through VkFFTLaunchParams struct
VKFFT_ERROR_EMPTY_tempBufferSize = 2005,	// Same error as VKFFT_ERROR_EMPTY_bufferSize if userTempBuffer is enabled
VKFFT_ERROR_EMPTY_tempBuffer = 2006,	// Same error as VKFFT_ERROR_EMPTY_buffer if userTempBuffer is enabled
VKFFT_ERROR_EMPTY_inputBufferSize = 2007,	// Same error as VKFFT_ERROR_EMPTY_bufferSize if isInputFormatted is enabled
VKFFT_ERROR_EMPTY_inputBuffer = 2008,	// Same error as VKFFT_ERROR_EMPTY_buffer if isInputFormatted is enabled
VKFFT_ERROR_EMPTY_outputBufferSize = 2009,	// Same error as VKFFT_ERROR_EMPTY_bufferSize if isOutputFormatted is enabled
VKFFT_ERROR_EMPTY_outputBuffer = 2010,	// Same error as VKFFT_ERROR_EMPTY_buffer if isOutputFormatted is enabled
VKFFT_ERROR_EMPTY_kernelSize = 2011,	// Same error as VKFFT_ERROR_EMPTY_bufferSize if performConvolution is enabled
VKFFT_ERROR_EMPTY_kernel = 2012,	// Same error as VKFFT_ERROR_EMPTY_buffer if performConvolution is enabled
VKFFT_ERROR_UNSUPPORTED_RADIX = 3001,	// VkFFT has encountered unsupported radix (more than 13) during decomposition and Bluestein's FFT fallback did not work
VKFFT_ERROR_UNSUPPORTED_FFT_LENGTH = 3002,	// VkFFT can not do this sequence length currently - it requires mor than three-upload Four step FFT
VKFFT_ERROR_UNSUPPORTED_FFT_LENGTH_R2C = 3003,	// VkFFT can not do this sequence length currently - odd multi-upload R2C/C2R FFTs
VKFFT_ERROR_UNSUPPORTED_FFT_LENGTH_DCT = 3004,	// VkFFT can not do this sequence length currently - multi-upload R2R transforms, odd DCT-IV transforms
VKFFT_ERROR_UNSUPPORTED_FFT_OMIT = 3005,	// VkFFT can not omit sequences in convolution calculations and R2C/C2R case
VKFFT_ERROR_FAILED_TO_ALLOCATE = 4001,	// VkFFT failed to allocate GPU memory
VKFFT_ERROR_FAILED_TO_MAP_MEMORY = 4002,	// 4002-4052 are handlers for errors of used backend APIs. They may indicate a driver failure. If they are thrown - report to the GitHub repo
VKFFT_ERROR_FAILED_TO_ALLOCATE_COMMAND_BUFFERS = 4003,
VKFFT_ERROR_FAILED_TO_BEGIN_COMMAND_BUFFER = 4004,
VKFFT_ERROR_FAILED_TO_END_COMMAND_BUFFER = 4005,
VKFFT_ERROR_FAILED_TO_SUBMIT_QUEUE = 4006,
VKFFT_ERROR_FAILED_TO_WAIT_FOR_FENCES = 4007,
VKFFT_ERROR_FAILED_TO_RESET_FENCES = 4008,
VKFFT_ERROR_FAILED_TO_CREATE_DESCRIPTOR_POOL = 4009,
VKFFT_ERROR_FAILED_TO_CREATE_DESCRIPTOR_SET_LAYOUT = 4010,
VKFFT_ERROR_FAILED_TO_ALLOCATE_DESCRIPTOR_SETS = 4011,
VKFFT_ERROR_FAILED_TO_CREATE_PIPELINE_LAYOUT = 4012,
VKFFT_ERROR_FAILED_SHADER_PREPROCESS = 4013,
VKFFT_ERROR_FAILED_SHADER_PARSE = 4014,
VKFFT_ERROR_FAILED_SHADER_LINK = 4015,
VKFFT_ERROR_FAILED_SPIRV_GENERATE = 4016,
VKFFT_ERROR_FAILED_TO_CREATE_SHADER_MODULE = 4017,
VKFFT_ERROR_FAILED_TO_CREATE_INSTANCE = 4018,
VKFFT_ERROR_FAILED_TO_SETUP_DEBUG_MESSENGER = 4019,
VKFFT_ERROR_FAILED_TO_FIND_PHYSICAL_DEVICE = 4020,
VKFFT_ERROR_FAILED_TO_CREATE_DEVICE = 4021,
VKFFT_ERROR_FAILED_TO_CREATE_FENCE = 4022,
VKFFT_ERROR_FAILED_TO_CREATE_COMMAND_POOL = 4023,
VKFFT_ERROR_FAILED_TO_CREATE_BUFFER = 4024,
VKFFT_ERROR_FAILED_TO_ALLOCATE_MEMORY = 4025,
VKFFT_ERROR_FAILED_TO_BIND_BUFFER_MEMORY = 4026,
VKFFT_ERROR_FAILED_TO_FIND_MEMORY = 4027,
VKFFT_ERROR_FAILED_TO_SYNCHRONIZE = 4028,
VKFFT_ERROR_FAILED_TO_COPY = 4029,
VKFFT_ERROR_FAILED_TO_CREATE_PROGRAM = 4030,
VKFFT_ERROR_FAILED_TO_COMPILE_PROGRAM = 4031, 
VKFFT_ERROR_FAILED_TO_GET_CODE_SIZE = 4032,
VKFFT_ERROR_FAILED_TO_GET_CODE = 4033,
VKFFT_ERROR_FAILED_TO_DESTROY_PROGRAM = 4034,
VKFFT_ERROR_FAILED_TO_LOAD_MODULE = 4035,
VKFFT_ERROR_FAILED_TO_GET_FUNCTION = 4036,
VKFFT_ERROR_FAILED_TO_SET_DYNAMIC_SHARED_MEMORY = 4037,
VKFFT_ERROR_FAILED_TO_MODULE_GET_GLOBAL = 4038,
VKFFT_ERROR_FAILED_TO_LAUNCH_KERNEL = 4039,
VKFFT_ERROR_FAILED_TO_EVENT_RECORD = 4040,
VKFFT_ERROR_FAILED_TO_ADD_NAME_EXPRESSION = 4041,
VKFFT_ERROR_FAILED_TO_INITIALIZE = 4042,
VKFFT_ERROR_FAILED_TO_SET_DEVICE_ID = 4043,
VKFFT_ERROR_FAILED_TO_GET_DEVICE = 4044,
VKFFT_ERROR_FAILED_TO_CREATE_CONTEXT = 4045,
VKFFT_ERROR_FAILED_TO_CREATE_PIPELINE = 4046,
VKFFT_ERROR_FAILED_TO_SET_KERNEL_ARG = 4047,
VKFFT_ERROR_FAILED_TO_CREATE_COMMAND_QUEUE = 4048,
VKFFT_ERROR_FAILED_TO_RELEASE_COMMAND_QUEUE = 4049,
VKFFT_ERROR_FAILED_TO_ENUMERATE_DEVICES = 4050,
VKFFT_ERROR_FAILED_TO_GET_ATTRIBUTE = 4051,
VKFFT_ERROR_FAILED_TO_CREATE_EVENT = 4052,
VKFFT_ERROR_FAILED_TO_CREATE_COMMAND_LIST = 4053,
VKFFT_ERROR_FAILED_TO_DESTROY_COMMAND_LIST = 4054,
VKFFT_ERROR_FAILED_TO_SUBMIT_BARRIER = 4055
} VkFFTResult;
\end{minted}
\end{mdframed}

\subsection{VkFFT application management functions}

VkFFT has a unified plan management model - all different transform
types/ dimensionalities/ precision use the same calls with configuration
done through VkFFTConfiguration struct. This section shows how to
initialize/use/free VkFFT with this unified model, while the next
one will go into how to configure VkFFTConfiguration correctly. All
of the functions operate on VkFFTApplication and VkFFTConfiguration
assuming they have been zero-initialized before usage, so do not forget
to do this when initializing:

\begin{mdframed}[backgroundcolor=bg]
\begin{minted}[tabsize=4,obeytabs,breaklines]{C}
VkFFTConfiguration configuration = {};
VkFFTApplication app = {};
\end{minted}
\end{mdframed}

\subsubsection{Function initializeVkFFT()}

\begin{mdframed}[backgroundcolor=bg]
\begin{minted}[tabsize=4,obeytabs,breaklines]{C}
VkFFTResult initializeVkFFT(VkFFTApplication* app, VkFFTConfiguration inputLaunchConfiguration)
\end{minted}
\end{mdframed}

Creates an FFT application (collection of forward and inverse plans).
As forward and inverse FFTs may have different memory layouts, can
have different normalizations - they are done as separate internal
plans inside VkFFTApplication. This call assumes the application to
be zero-initialized, so can be only done once on a particular application,
until it is deleted.

If the initializeVkFFT call fails, it frees all allocated by VkFFT
CPU/GPU resources and sets the application to zero. VkFFTResult is
returned with an error code corresponding to what went wrong.

In case of success, VkFFTApplication will contain initialized plans
with compiled kernels ready for execution with VKFFT\_SUCCESS returned.

\subsubsection{Function VkFFTAppend()}

\begin{mdframed}[backgroundcolor=bg]
\begin{minted}[tabsize=4,obeytabs,breaklines]{C}
VkFFTResult VkFFTAppend(VkFFTApplication* app, int inverse, VkFFTLaunchParams* launchParams)
\end{minted}
\end{mdframed}

Performs FFT in the int inverse direction (-1 for forward FFT, 1 for
inverse FFT). FFT plans are selected from the VkFFTApplication collection
automatically. VkFFTApplication must be initialized with initializeVkFFT
call before. VkFFTLaunchParams struct allows for pre-launch configuration
of some parameters, namely:
\begin{itemize}
\item buffer - similar to how FFTW/cuFFT expects input/output data pointers
in {*}execC2C (and other) function calls, VkFFT allows specifying
memory used for computations at launch. It must have the same size/layout/strides
as defined during the application creation.
\item inputBuffer/outputBuffer/tempBuffer/kernel - other buffers can also
be specified at launch. In addition to them having the same size/layout/strides
as defined during the application creation, the application must be
created with flags enabling the corresponding buffer usage: isInputFormatted/isOutputFormatted/userTempBuffer/performConvolution
respectively.
\item bufferOffset/tempBufferOffset/inputBufferOffset/outputBufferOffset/kernelOffset
- specify if VkFFT has to offset the first element position inside
the corresponding buffer. In bytes. Default 0. specifyOffsetsAtLaunch
parameter must be enabled during the initializeVkFFT call before. 
\end{itemize}
Depending on the API, the execution model may vary and require additional
information at launch:
\begin{itemize}
\item Vulkan API: VkFFT appends a sequence of vkCmdDispatch calls to the
user-defined VkCommandBuffer (with respective push constants/descriptor
sets/pipelines/memory barriers bindings). VkCommandBuffer must be
provided as a pointer in VkFFTLaunchParams. VkCommandBuffer must be
in the writing stage, started with vkBeginCommandBuffer call. After
VkFFTAppend has finished, provided VkCommandBuffer will contain a
sequence of operations performing FFT. The first call of the sequence
has no input memory barrier, the last call has one, ensuring FFT has
finished execution.
\item CUDA/HIP API: if the user wants to use streams, they have to be provided
during the application configuration stage. VkFFTAppend performs a
series of cuLaunchKernel, which are sequential if appended to one
stream and synchronized if appended to multiple streams.
\item OpenCL API: similar to Vulkan, VkFFT appends a sequence of clEnqueueNDRangeKernel
calls to user-defined cl\_command\_queue. Currently, they are all
assumed to be sequential. cl\_command\_queue must be provided as a
pointer in VkFFTLaunchParams. 
\item Level Zero API: similar to Vulkan, VkFFT appends a sequence of zeCommandListAppendLaunchKernel
calls to user-defined command list ze\_command\_list\_handle\_t. They
have execution barriers between. ze\_command\_list\_handle\_t must
be provided as a pointer in VkFFTLaunchParams. 
\end{itemize}
If VkFFT fails during the VkFFTAppend call, it will not free the application
and allocated there resources - use a separate call for that.

\subsubsection{Function deleteVkFFT()}

\begin{mdframed}[backgroundcolor=bg]
\begin{minted}[tabsize=4,obeytabs,breaklines]{C}
void deleteVkFFT(VkFFTApplication* app)
\end{minted}
\end{mdframed}

Performs deallocation of resources used in the provided application.
Returns application to the zero-initialized state.

\subsubsection{Function VkFFTGetVersion()}

\begin{mdframed}[backgroundcolor=bg]
\begin{minted}[tabsize=4,obeytabs,breaklines]{C}
int VkFFTGetVersion()
\end{minted}
\end{mdframed}

Returns the version of the VkFFT library in the X.XX.XX format (without
dots).

\subsection{VkFFT configuration}

This section will cover all the parameters that can be specified in
the VkFFTConfiguration struct. It will start with a short description
of the struct (intended to be used as a cheat sheet), then go for
each field in detail.

\begin{mdframed}[backgroundcolor=bg]
\begin{minted}[tabsize=4,obeytabs,breaklines]{C}
typedef struct {
// Required parameters: 	
uint64_t FFTdim;	// FFT dimensionality (1, 2 or 3)
uint64_t size[3];	// WHD - system dimensions
#if(VKFFT_BACKEND==0) //Vulkan API
VkPhysicalDevice* physicalDevice;	// Pointer to Vulkan physical device, obtained from vkEnumeratePhysicalDevices
VkDevice* device;	// Pointer to Vulkan device, created with vkCreateDevice
VkQueue* queue;	// Pointer to Vulkan queue, created with vkGetDeviceQueue
VkCommandPool* commandPool;	// Pointer to Vulkan command pool, created with vkCreateCommandPool
VkFence* fence;	// Pointer to Vulkan fence, created with vkCreateFence
uint64_t isCompilerInitialized;	// Specify if glslang compiler has been intialized before (0 - off, 1 - on). Default 0 
#elif(VKFFT_BACKEND==1) //CUDA API
CUdevice* device;	// Pointer to CUDA device, obtained from cuDeviceGet 	
cudaStream_t* stream;	// Pointer to streams (can be more than 1), where to execute the kernels. Deafult 0
uint64_t num_streams;	// Try to submit CUDA kernels in multiple streams for asynchronous execution. Default 1 
#elif(VKFFT_BACKEND==2) //HIP API
hipDevice_t* device;	// Pointer to HIP device, obtained from hipDeviceGet
hipStream_t* stream;	// Pointer to streams (can be more than 1), where to execute the kernels. Deafult 0
uint64_t num_streams;	// Try to submit HIP kernels in multiple streams for asynchronous execution. Default 1 
#elif(VKFFT_BACKEND==3) //OpenCL API
cl_platform_id* platform;	// NOT REQUIRED
cl_device_id* device;	// Pointer to OpenCL device, obtained from clGetDeviceIDs
cl_context* context;	// Pointer to OpenCL context, obtained from clCreateContext
#elif(VKFFT_BACKEND==4) //Level Zero API
ze_device_handle_t* device;	// Pointer to Level Zero device, obtained from zeDeviceGet
ze_context_handle_t* context;	// Pointer to Level Zero context, obtained from zeContextCreate
ze_command_queue_handle_t* commandQueue;	// Pointer to Level Zero command queue with compute and copy capabilities, obtained from zeCommandQueueCreate
uint32_t commandQueueID;	// ID of the commandQueue with compute and copy capabilities
#endif

// Data parameters (buffers can be specified at launch):
uint64_t userTempBuffer;	// Buffer allocated by app automatically if needed to reorder Four step algorithm. Setting to non zero value enables manual user allocation (0 - off, 1 - on)
uint64_t bufferNum;	// Multiple buffer sequence storage is Vulkan only. Default 1
uint64_t tempBufferNum;	// Multiple buffer sequence storage is Vulkan only. Default 1, buffer allocated by app automatically if needed to reorder Four step algorithm. Setting to non zero value enables manual user allocation 	
uint64_t inputBufferNum;	// Multiple buffer sequence storage is Vulkan only. Default 1, if isInputFormatted is enabled 
uint64_t outputBufferNum;	// Multiple buffer sequence storage is Vulkan only. Default 1, if isOutputFormatted is enabled 
uint64_t kernelNum;	// Multiple buffer sequence storage is Vulkan only. Default 1, if performConvolution is enabled
uint64_t* bufferSize;	// Array of buffers sizes in bytes
uint64_t* tempBufferSize;	// Array of temp buffers sizes in bytes. Default set to bufferSize sum, buffer allocated by app automatically if needed to reorder Four step algorithm. Setting to non zero value enables manual user allocation 
uint64_t* inputBufferSize;	// Array of input buffers sizes in bytes, if isInputFormatted is enabled
uint64_t* outputBufferSize;	// Array of output buffers sizes in bytes, if isOutputFormatted is enabled
uint64_t* kernelSize;	// Array of kernel buffers sizes in bytes, if performConvolution is enabled
#if(VKFFT_BACKEND==0) //Vulkan API
VkBuffer* buffer;	// Pointer to array of buffers (or one buffer) used for computations
VkBuffer* tempBuffer;	// Needed if reorderFourStep is enabled to transpose the array. Same sum size or bigger as buffer (can be split in multiple). Default 0. Setting to non zero value enables manual user allocation
VkBuffer* inputBuffer;	// Pointer to array of input buffers (or one buffer) used to read data from if isInputFormatted is enabled
VkBuffer* outputBuffer;	// Pointer to array of output buffers (or one buffer) used to write data to if isOutputFormatted is enabled
VkBuffer* kernel;	// Pointer to array of kernel buffers (or one buffer) used to read kernel data from if performConvolution is enabled
#elif(VKFFT_BACKEND==1) //CUDA API
void** buffer;	// Pointer to device buffer used for computations
void** tempBuffer;	// Needed if reorderFourStep is enabled to transpose the array. Same size as buffer. Default 0. Setting to non zero value enables manual user allocation 
void** inputBuffer;	// Pointer to device buffer used to read data from if isInputFormatted is enabled
void** outputBuffer;	// Pointer to device buffer used to write data to if isOutputFormatted is enabled
void** kernel;	// Pointer to device buffer used to read kernel data from if performConvolution is enabled 
#elif(VKFFT_BACKEND==2) //HIP API
void** buffer;	// Pointer to device buffer used for computations
void** tempBuffer;	// Needed if reorderFourStep is enabled to transpose the array. Same size as buffer. Default 0. Setting to non zero value enables manual user allocation
void** inputBuffer;	// Pointer to device buffer used to read data from if isInputFormatted is enabled
void** outputBuffer;	// Pointer to device buffer used to write data to if isOutputFormatted is enabled
void** kernel;	// Pointer to device buffer used to read kernel data from if performConvolution is enabled 
#elif(VKFFT_BACKEND==3) //OpenCL API
cl_mem* buffer;	// Pointer to device buffer used for computations
cl_mem* tempBuffer;	// Needed if reorderFourStep is enabled to transpose the array. Same size as buffer. Default 0. Setting to non zero value enables manual user allocation
cl_mem* inputBuffer;	// Pointer to device buffer used to read data from if isInputFormatted is enabled
cl_mem* outputBuffer;	// Pointer to device buffer used to write data to if isOutputFormatted is enabled
cl_mem* kernel;	// Pointer to device buffer used to read kernel data from if performConvolution is enabled
#endif
uint64_t bufferOffset;	// Specify if VkFFT has to offset the first element position inside the buffer. In bytes. Default 0
uint64_t tempBufferOffset;	// Specify if VkFFT has to offset the first element position inside the temp buffer. In bytes. Default 0
uint64_t inputBufferOffset;	// Specify if VkFFT has to offset the first element position inside the input buffer. In bytes. Default 0
uint64_t outputBufferOffset;	// Specify if VkFFT has to offset the first element position inside the output buffer. In bytes. Default 0
uint64_t kernelOffset;	// Specify if VkFFT has to offset the first element position inside the kernel. In bytes. Default 0
uint64_t specifyOffsetsAtLaunch;	// Specify if offsets will be selected with launch parameters VkFFTLaunchParams (0 - off, 1 - on). Default 0

// Optional: (default 0 if not stated otherwise)
uint64_t coalescedMemory;	// In bytes, for Nvidia and AMD is equal to 32, Intel is equal 64, scaled for half precision. Going to work regardless, but if specified by user correctly, the performance will be higher.
uint64_t aimThreads;	// Aim at this many threads per block. Default 128
uint64_t numSharedBanks;	// How many banks shared memory has. Default 32
uint64_t inverseReturnToInputBuffer;	// return data to the input buffer in inverse transform (0 - off, 1 - on). isInputFormatted must be enabled
uint64_t numberBatches;	// N - used to perform multiple batches of initial data. Default 1
uint64_t useUint64;	// Use 64-bit addressing mode in generated kernels
uint64_t omitDimension[3];	// Disable FFT for this dimension (0 - FFT enabled, 1 - FFT disabled). Default 0. Doesn't work for R2C for now. Doesn't work with convolutions.
uint64_t fixMaxRadixBluestein;	// Controls the padding of sequences in Bluestein convolution. If specified, padded sequence will be made of up to fixMaxRadixBluestein primes. Default: 2 for up to 1048576 combined dimension FFT system, 7 after. Min = 2, Max = 13.
uint64_t performBandwidthBoost; // Try to reduce coalsesced number by a factor of X to get bigger sequence in one upload for strided axes. Default: -1 for DCT, 2 for Bluestein's algorithm (or -1 if DCT), 0 otherwise  
uint64_t doublePrecision;	// Perform calculations in double precision (0 - off, 1 - on).
uint64_t halfPrecision;	// Perform calculations in half precision (0 - off, 1 - on)
uint64_t halfPrecisionMemoryOnly;	// Use half precision only as input/output buffer. Input/Output have to be allocated as half, buffer/tempBuffer have to be allocated as float (out-of-place mode only). Specify isInputFormatted and isOutputFormatted to use (0 - off, 1 - on)
uint64_t doublePrecisionFloatMemory;	// Use FP64 precision for all calculations, while all memory storage is done in FP32.
uint64_t performR2C;	// Perform R2C/C2R decomposition (0 - off, 1 - on)
uint64_t performDCT;	// Perform DCT transformation (X - DCT type, 1-4)
uint64_t disableMergeSequencesR2C;	// Disable merging of two real sequences to reduce calculations (0 - off, 1 - on) 
uint64_t normalize;	// Normalize inverse transform (0 - off, 1 - on)
uint64_t disableReorderFourStep;	// Disables unshuffling of Four step algorithm. Requires tempbuffer allocation (0 - off, 1 - on)
uint64_t useLUT;	// Switches from calculating sincos to using precomputed LUT tables (0 - off, 1 - on). Configured by initialization routine
uint64_t makeForwardPlanOnly;	// Generate code only for forward FFT (0 - off, 1 - on)
uint64_t makeInversePlanOnly;	// Generate code only for inverse FFT (0 - off, 1 - on)
uint64_t bufferStride[3];	// Buffer strides - default set to x - x*y - x*y*z values
uint64_t isInputFormatted;	// Specify if input buffer is padded - 0 - padded, 1 - not padded. For example if it is not padded for R2C if out-of-place mode is selected (only if numberBatches==1 and numberKernels==1)
uint64_t isOutputFormatted;	// Specify if output buffer is padded - 0 - padded, 1 - not padded. For example if it is not padded for R2C if out-of-place mode is selected (only if numberBatches==1 and numberKernels==1)
uint64_t inputBufferStride[3];	// Input buffer strides. Used if isInputFormatted is enabled. Default set to bufferStride values
uint64_t outputBufferStride[3];	// Output buffer strides. Used if isInputFormatted is enabled. Default set to bufferStride values
uint64_t considerAllAxesStrided;	// Will create plan for non-strided axis similar as a strided axis - used with disableReorderFourStep to get the same layout for Bluestein kernel (0 - off, 1 - on)
uint64_t keepShaderCode;	// Will keep shader code and print all executed shaders during the plan execution in order (0 - off, 1 - on)
uint64_t printMemoryLayout;	// Will print order of buffers used in shaders (0 - off, 1 - on) 

uint64_t saveApplicationToString;	// Will save all compiled binaries to VkFFTApplication.saveApplicationString (will be allocated by VkFFT, deallocated with deleteVkFFT call). VkFFTApplication.applicationStringSize will contain size of binary in bytes. (0 - off, 1 - on)
uint64_t loadApplicationFromString;	// Will load all binaries from loadApplicationString instead of recompiling them (must be allocated by user, must contain what saveApplicationToString call generated previously in VkFFTApplication.saveApplicationString). (0 - off, 1 - on). Mutually exclusive with saveApplicationToString
void* loadApplicationString;	// Memory array (uint32_t* for Vulkan/HIP, char* for CUDA/OpenCL) through which user can load VkFFT binaries, must be provided by user if loadApplicationFromString = 1. 

// Optional zero padding control parameters: (default 0 if not stated otherwise)
uint64_t performZeropadding[3];	// Don't read some data/perform computations if some input sequences are zeropadded for each axis (0 - off, 1 - on)
uint64_t fft_zeropad_left[3];	// Specify start boundary of zero block in the system for each axis
uint64_t fft_zeropad_right[3];	// Specify end boundary of zero block in the system for each axis
uint64_t frequencyZeroPadding;	// Set to 1 if zeropadding of frequency domain, default 0 - spatial zeropadding

// Optional convolution control parameters: (default 0 if not stated otherwise)
uint64_t performConvolution;	// Perform convolution in this application (0 - off, 1 - on). Disables reorderFourStep parameter
uint64_t coordinateFeatures;	// C - coordinate, or dimension of features vector. In matrix convolution - size of a vector
uint64_t matrixConvolution;	// If equal to 2 perform 2x2, if equal to 3 perform 3x3 matrix-vector convolution. Overrides coordinateFeatures
uint64_t symmetricKernel;	// Specify if kernel in 2x2 or 3x3 matrix convolution is symmetric
uint64_t numberKernels;	// N - only used in convolution step - specify how many kernels were initialized before. Expands one input to multiple (batched) output
uint64_t kernelConvolution;	// Specify if this application is used to create kernel for convolution, so it has the same properties. performConvolution has to be set to 0 for kernel creation

// Register overutilization (experimental): (default 0 if not stated otherwise)
uint64_t registerBoost;	// Specify if register file size is bigger than shared memory and can be used to extend it X times (on Nvidia 256KB register file can be used instead of 32KB of shared memory, set this constant to 4 to emulate 128KB of shared memory). Defaults: Nvidia - 4 in Vulkan/OpenCL, 1 in CUDA backend; AMD - 2 if shared memory >= 64KB, else 4 in Vulkan/OpenCL backend, 1 in HIP backend; Intel - 1 if shared memory >= 64KB, else 2 in Vulkan/OpenCL/Level Zero backends; Default 1
uint64_t registerBoostNonPow2;	// Specify if register overutilization should be used on non power of 2 sequences (0 - off, 1 - on)
uint64_t registerBoost4Step;	// Specify if register file overutilization should be used in big sequences (>2^14), same definition as registerBoost. Default 1
//not used techniques:
uint64_t swapTo3Stage4Step;	// Specify at which power of 2 to switch from 2 upload to 3 upload 4-step FFT, in case if making max sequence size lower than coalesced sequence helps to combat TLB misses. Default 0 - disabled. Must be at least 17
uint64_t devicePageSize;	// In KB, the size of a page on the GPU. Setting to 0 disables local buffer split in pages
uint64_t localPageSize;	// In KB, the size to split page into if sequence spans multiple devicePageSize pages

// Automatically filled based on device info (still can be reconfigured by user):
uint64_t maxComputeWorkGroupCount[3];	// maxComputeWorkGroupCount from VkPhysicalDeviceLimits
uint64_t maxComputeWorkGroupSize[3];	// maxComputeWorkGroupCount from VkPhysicalDeviceLimits
uint64_t maxThreadsNum;	// Max number of threads from VkPhysicalDeviceLimits
uint64_t sharedMemorySizeStatic;	// Available for static allocation shared memory size, in bytes
uint64_t sharedMemorySize;	// Available for allocation shared memory size, in bytes
uint64_t sharedMemorySizePow2;	// Power of 2 which is less or equal to sharedMemorySize, in bytes
uint64_t warpSize;	// Number of threads per warp/wavefront.
uint64_t halfThreads;	// Intel fix
uint64_t allocateTempBuffer;	// Buffer allocated by app automatically if needed to reorder Four step algorithm. Parameter to check if it has been allocated
uint64_t reorderFourStep;	// Unshuffle Four step algorithm. Requires tempbuffer allocation (0 - off, 1 - on). Default 1.
int64_t maxCodeLength;	// Specify how big can be buffer used for code generation (in char). Default 1000000 chars. 
int64_t maxTempLength;	// Specify how big can be buffer used for intermediate string sprintfs be (in char). Default 5000 chars. If code segfaults for some reason - try increasing this number.
#if(VKFFT_BACKEND==0) //Vulkan API
VkDeviceMemory tempBufferDeviceMemory;	// Filled at app creation
VkCommandBuffer* commandBuffer;	// Filled at app execution
VkMemoryBarrier* memory_barrier;	// Filled at app creation
#elif(VKFFT_BACKEND==1) //CUDA API
cudaEvent_t* stream_event;	// Filled at app creation
uint64_t streamCounter;	// Filled at app creation
uint64_t streamID;	// Filled at app creation
#elif(VKFFT_BACKEND==2) //HIP API
hipEvent_t* stream_event;	// Filled at app creation
uint64_t streamCounter;	// Filled at app creation
uint64_t streamID;	// Filled at app creation
#elif(VKFFT_BACKEND==3) //OpenCL API
cl_command_queue* commandQueue;	// Filled at app creation
#elif(VKFFT_BACKEND==4)
ze_command_list_handle_t* commandList;	// Filled at app creation
#endif
} VkFFTConfiguration;
\end{minted}
\end{mdframed}

\subsubsection{Driver API parameters}

In order to work, VkFFT needs some structures that are provided by
the driver. They are backend API-dependent. VkFFT will return corresponding
VkFFTResult if one of these structures are not provided (value equal
to zero) unless it is stated that there is a default value assigned.
VkFFT will not modify provided values directly.

Vulkan API will need the following information:
\begin{itemize}
\item VkPhysicalDevice{*} physicalDevice - Pointer to Vulkan physical device,
obtained from vkEnumeratePhysicalDevices()
\item VkDevice{*} device - Pointer to Vulkan device, created with vkCreateDevice()
\item VkQueue{*} queue - Pointer to Vulkan queue, created with vkGetDeviceQueue()
\item VkCommandPool{*} commandPool - Pointer to Vulkan command pool, created
with vkCreateCommandPool()
\item VkFence{*} fence - Pointer to Vulkan fence, created with vkCreateFence()
\item uint64\_t isCompilerInitialized - Specify if glslang compiler has
been intialized before (0 - off, 1 - on). Default 0 - VkFFT will call
glslang\_initialize\_process() at initializeVkFFT() and glslang\_finialize\_process()
at deleteVkFFT() calls.
\end{itemize}
CUDA API will need the following information:
\begin{itemize}
\item CUdevice{*} device - Pointer to CUDA device, obtained from cuDeviceGet()
\item cudaStream\_t{*} stream - Pointer to streams (can be more than 1),
where to execute the kernels. Default 0. Streams must be associated
with the provided device. There is no real benefit in having more
than one, however. 
\item uint64\_t num\_streams - Try to submit CUDA kernels in multiple streams
for asynchronous execution. Default 1 
\end{itemize}
HIP API will need the following information:
\begin{itemize}
\item hipDevice\_t{*} device - Pointer to HIP device, obtained from hipDeviceGet()
\item hipStream\_t{*} stream - Pointer to streams (can be more than 1),
where to execute the kernels. Default 0. Streams must be associated
with the provided device. There is no real benefit in having more
than one, however. 
\item uint64\_t num\_streams - Try to submit HIP kernels in multiple streams
for asynchronous execution. Default 1 
\end{itemize}
OpenCL API will need the following information:
\begin{itemize}
\item cl\_device\_id{*} device - Pointer to OpenCL device, obtained from
clGetDeviceIDs()
\item cl\_context{*} context - Pointer to OpenCL context, obtained from
clCreateContext()
\end{itemize}
Level Zero API will need the following information:
\begin{itemize}
\item ze\_device\_handle\_t{*} device - Pointer to Level Zero device, obtained
from zeDeviceGet()
\item ze\_context\_handle\_t{*} context - Pointer to Level Zero context,
obtained from zeContextGet()
\item ze\_command\_queue\_handle\_t{*} commandQueue - Pointer to Level Zero
command queuewith compute and copy capabilities, obtained from zeCommandQueueCreate()
\item uint32\_t commandQueueID - ID of the commandQueue with compute and
copy capabilities
\end{itemize}

\subsubsection{Memory management parameters}

There are five buffer types user can provide to VkFFT: 
\begin{itemize}
\item the main buffer (buffer)
\item temporary buffer used for calculations requiring out-of-place writes
(tempBuffer)
\item separate input buffer, from which initial read is performed (inputBuffer)
\item separate output buffer, to which final write is performed (outputBuffer)
\item kernel buffer, used for calculation of convolutions and cross-correlations
(kernel)
\end{itemize}
These buffers must be passed by a pointer: in Vulkan API they are
provided as VkBuffer{*}, in CUDA, HIP and Level Zero they are provided
as void{*}{*}, in OpenCL, they are provided as cl\_mem{*}. Even though
the underlying structure (VkBuffer, void{*}, cl\_mem) is not a memory
but just a number that the driver can use to access corresponding
allocated memory on the GPU, passing them by a pointer allows for
the user to query multiple GPU allocated buffers for VkFFT to use.
Currently, it is only supported in Vulkan API - each of five buffer
types can be made out of multiple separate memory allocations. For
example, it is possible to combine multiple small unused at the point
of FFT calculation buffers to form a tempBuffer. This option also
allows Vulkan API to overcome the limit of 4GB for a single memory
allocation - due to the fact that Vulkan can only use 32-bit numbers
for addressing (other APIs support 64-bit addressing). 

To use the buffers other than the main buffer, the user has to specify
this in configuration at the application creation stage (set to zero
by default, optional parameters): 
\begin{itemize}
\item uint64\_t userTempBuffer - enables manual temporary buffer allocation
(otherwise it is managed by VkFFT)
\item uint64\_t isInputFormatted - specifies that initial read is performed
from a separate buffer (inputBuffer)
\item uint64\_t isOutputFormatted - specifies that final write is performed
to a separate buffer (outputBuffer)
\item uint64\_t performConvolution - enables convolution calculations, which
requires precomputed kernel (kernel)
\end{itemize}
Buffer sizes (bufferSize/tempBufferSize/inputBufferSize/outputBufferSize/kernelSize)
are provided as a uint64\_t pointer to an array, where each element
corresponds to the buffer size of the buffer with the same placement
in the buffer array. Buffer sizes have to be provided in Vulkan API
(due to the stricter memory management model and multiple buffer support)
and are optional in other backends (they can be useful to determine
when to switch for 64-bit addressing).

Buffer number (bufferNum/tempBufferNum/inputBufferNum/outputBufferNum/kernelNum)
corresponds to how many elements are in the buffer and buffer size
array. By default it is set to 1 and is not required to be provided
by the user. Non-Vulkan backends currently don't support values other
than default. Optional parameter.

Buffer offset (bufferOffset/ tempBufferOffset/ inputBufferOffset/
outputBufferOffset/ kernelOffset) specifies offset from the start
of the buffer sequence. It must be specified in bytes and must be
divisible by the number type size used in the corresponding array
(otherwise, the offset will be truncated). It is provided as a single
uint64\_t value. Can be provided at launch time, if specifyOffsetsAtLaunch
parameter is enabled during initialization call. Optional parameters.

User can provide custom dimension strides for buffer/inputBuffer/outputBuffer
buffers - uint64\_t{[}3{]} array. Strides are specified in elements
used in the array (not bytes). The first element corresponds to the
stride between elements in the H direction, the second corresponds
to the D direction and the third to C (or N, if the number of elements
in C is 1). The first axis is assumed to be non-strided. Must be at
least of the same size as default strides, otherwise the behavior
is undefined. Optional parameters.

uint64\_t inverseReturnToInputBuffer - an option that allows setting
the final output buffer of the inverse transform to the same buffer,
initial read of forward transform is performed from (inputBuffer,
if isInputFormatted enabled). Optional parameter.

\subsubsection{General FFT parameters }

This section describes part of the configuration structure responsible
for FFT specification. 

uint64\_t FFTdim - dimensionality of the transform (1, 2 or 3). Required
parameter.

uint64\_t size{[}3{]} - WHD dimensions of the transform. Required
parameter.

uint64\_t numberBatches - N parameter of the transform. By default,
it is set to 1. Optional parameter.

uint64\_t performR2C - perform R2C/C2R decomposition. performDCT must
be set to 0. Default 0, set to 1 to enable. Optional parameter.

uint64\_t performDCT - perform DCT transformation. performR2C must
be set to 0. Default 0, set to X for DCT-X (currently supported X:
1, 2, 3 and 4). Optional parameter.

uint64\_t normalize - enabling this parameter will make the inverse
transform divide the result by the FFT length. Default 0, set to 1
to enable. Optional parameter.

\subsubsection{Precision parameters (and some things that can affect it):}

uint64\_t doublePrecision - perform calculations in double precision.
Default 0, set to 1 to enable. In Vulkan/OpenCL/Level Zero your device
must support double-precision functionality. Optional parameter.

uint64\_t doublePrecisionFloatMemory - perform calculations in double
precision, but all intermediate and final storage in float. Input/Output/main
buffers must have single-precision layout. doublePrecision must be
set to 0. This option increases precision, but not that much to be
recommended for actual use. Default 0, set to 1 to enable. In Vulkan/OpenCL/Level
Zero your device must support double-precision functionality. Experimental
feature. Optional parameter.

uint64\_t halfPrecision - half-precision in VkFFT is implemented only
as memory optimization. All calculations are done in single precision
(similar way as doublePrecisionFloatMemory works for double and single
precision). Default 0, set to 1 to enable. Works only in Vulkan API
now, experimental feature (half precision seems to have bad precision
for the first FFT element). Optional parameter.

uint64\_t halfPrecisionMemoryOnly - another way of performing half-precision
in VkFFT, it will use half-precision only for initial and final memory
storage in input/output buffer. Input/Output have to be allocated
as half, buffer/tempBuffer have to be allocated as float (out-of-place
mode only). Specify isInputFormatted and isOutputFormatted to use.
So, for example, intermediate storage between axes FFTs in the multidimensional
case will be done in single precision, as opposed to half-precision
in the base halfPrecision case. halfPrecision must be set to 1. Default
0, set to 1 to enable. Works only in Vulkan API now, experimental
feature. Optional parameter.

uint64\_t useLUT - switches from calculating sines and cosines (via
special function units in single precision or as a polynomial approximation
in double precision) to using precomputed Look-Up Tables. Default
0 in single precision, 1 in double precision, set to 1 to enable.
Set to 1 by default for Intel GPUs. If you have issues with single-precision
accuracy on your GPU, try enabling this parameter (mobile GPUs may
be affected). Optional parameter.

\subsubsection{Advanced parameters (code will work fine without using them)}

uint64\_t omitDimension{[}3{]} - parameter, that disables the FFT
calculation for a particular axis (WHD). Note, that omitted dimensions
still need to be included in FFTdim and size. This parameter simply
works as a switch during execution - by not executing the particular
dimension code. It doesn't work with the non-strided axis (W) of R2C/C2R
mode. It doesn't work with convolution calculations. Default 0, set
to 1 to enable. Optional parameter.

uint64\_t useUint64 - forces VkFFT to use 64-bit addressing in generated
kernels. It is automatically enabled if the estimated buffer size
is more than 4GB. Doesn't work with the Vulkan backend. By default,
it is set to 0. Optional parameter.

uint64\_t coalescedMemory - number of bytes to coalesce per one transaction.
For Nvidia and AMD is equal to 32, Intel is equal to 64. Going to
work regardless, but if specified by the user correctly, the performance
will be higher. Default 64 for other GPUs. For half-precision should
be multiplied by two. Should be a power of two. Optional parameter.

uint64\_t numSharedBanks - configure the number of shared banks on
the target GPU. Default 32. Minor performance boost as it solves shared
memory conflicts for the power of two systems. Optional parameter.

uint64\_t aimThreads - try to aim all kernels at this amount of threads.
Gains/losses are not predictable, just a parameter to play with (it
is not guaranteed that the target kernel will use that many threads).
Default 128. Optional parameter.

uint64\_t useUint64 - forces 64-bit addressing in generated kernels.
Should be enabled automatically for systems spanning more than 4GB,
but it is better to have an option to force it as a failsafe. Doesn't
work in Vulkan API (use multiple buffer binding). Default 0, set to
1 to enable. Optional parameter.

uint64\_t fixMaxRadixBluestein - controls the padding of sequences
in Bluestein convolution. If specified, the padded sequence will be
made of up to fixMaxRadixBluestein primes. Default: 2 for up to 1048576
combined dimension FFT system, 7 after. Min = 2, Max = 13. Optional
parameter.

uint64\_t performBandwidthBoost - try to reduce coalsesced number
by a factor of X to get bigger sequence in one upload for strided
axes. Default: -1(inf) for DCT, 2 for Bluestein's algorithm (or -1
if DCT), 0 otherwise 

uint64\_t disableMergeSequencesR2C - disable the optimization that
performs merging of two real sequences to reduce calculations (in
R2C/C2R and R2R). If enabled, calculations will be performed by simply
setting the imaginary component to zero. Default 0, set to 1 to enable.
Optional parameter.

uint64\_t disableReorderFourStep - disables unshuffling of the Four
Step FFT algorithm (last transposition of data). With this option
enabled, tempBuffer will not be needed (unless it is required by Bluestein's
multi-upload FFT algorithm). Default 0, set to 1 to enable. Automatically
enabled for convolution calculations and Bluestein's algorithm. Optional
parameter.

uint64\_t makeForwardPlanOnly - generate code only for forward FFT.
Default 0, set to 1 to enable. Mutually exclusive with makeInversePlanOnly.
Optional parameter.

uint64\_t makeInversePlanOnly - generate code only for inverse FFT.
Default 0, set to 1 to enable. Mutually exclusive with makeForwardPlan.
Optional parameter.

uint64\_t considerAllAxesStrided - will create a plan for a non-strided
axis similar to a strided axis (used with disableReorderFourStep to
get the same layout for Bluestein kernel). Default 0, set to 1 to
enable. Optional parameter.

uint64\_t keepShaderCode - debugging option, will keep shader code
and print all executed shaders during the plan execution in order.
Default 0, set to 1 to enable. Optional parameter.

uint64\_t printMemoryLayout - debugging option, will print order of
buffers used in kernels. Default 0, set to 1 to enable. Optional parameter. 

uint64\_t saveApplicationToString - will save all compiled binaries
to VkFFTApplication.saveApplicationString (will be allocated by VkFFT,
deallocated with deleteVkFFT call). VkFFTApplication.applicationStringSize
will contain size of binary in bytes. Default 0, set to 1 to enable.
Optional parameter.

uint64\_t loadApplicationFromString - will load all binaries from
loadApplicationString instead of recompiling them (loadApplicationString
must be allocated by user, must contain what saveApplicationToString
call generated previously in VkFFTApplication.saveApplicationString).
Default 0, set to 1 to enable. Optional parameter. Mutually exclusive
with saveApplicationToString 

void{*} loadApplicationString - memory array (uint32\_t{*} for Vulkan,
HIP and Level Zero, char{*} for CUDA/OpenCL) through which user can
load VkFFT binaries, must be provided by user if loadApplicationFromString
= 1. 

\subsubsection{Zero padding parameters}

uint64\_t performZeropadding{[}3{]} - do not read/write some data/perform
computations if some part of the sequence is known to have zeros.
Set separately for each axis (WHD). If enabled, all 1D sequences in
this direction will be considered padded (independent of other zero-padded
axes). Default 0, set to 1 to enable. Optional parameter. 

uint64\_t fft\_zeropad\_left{[}3{]} - specify start boundary of zero
block in the system for each axis. Default 0, set to the value between
0 and size{[}X{]}-1. Optional parameter.

uint64\_t fft\_zeropad\_right{[}3{]} - specify end boundary of zero
block in the system for each axis. Default 0, set to the value between
fft\_zeropad\_left{[}X{]} and size{[}X{]}-1. Optional parameter.

uint64\_t frequencyZeroPadding - enables zero padding of the frequency
domain, so the first read of inverse FFT will consider the parts of
the system from fft\_zeropad\_left to fft\_zeropad\_right as zero.
Default 0 - spatial zero padding, set to 1 to enable. Optional parameter. 

\subsubsection{Convolution parameters}

uint64\_t performConvolution - main parameter that enables convolutions
in the application. If enabled, you must specify kernel buffer, number
of kernel buffers and kernel sizes (in Vulkan API). Disables reordering
of the Four Step FFT algorithm. Default 0, set to 1 to enable. Optional
parameter. 

uint64\_t conjugateConvolution - default 0, set to 1 to enable enables
conjugation of the sequence FFT is currently done on, 2 to enable
conjugation of the convolution kernel. Optional parameter. 

uint64\_t crossPowerSpectrumNormalization - normalize the FFT {*}
kernel multiplication in frequency domain. Default 0, set to 1 to
enable. Optional parameter. 

uint64\_t coordinateFeatures - max coordinate (C), or dimension of
the features vector. In matrix convolution - the size of the vector.
The main purpose is to support Matrix-Vector convolutions. Use numberBatches
parameter in tasks, not requiring two separate coordinate-like enumerations
of data. Default 1. Optional parameter.

uint64\_t matrixConvolution - set to 2 to perform 2x2, set to 3 to
perform 3x3 matrix-vector convolution. Matrix-vector convolution is
a form of point-wise multiplication in the Fourier space, used by
the convolution theorem, where multiplication takes the form of Matrix-vector
multiplication. Overrides coordinateFeatures during execution. Default
0. Optional parameter. 

uint64\_t symmetricKernel - specify if kernel in 2x2 or 3x3 matrix
convolution is symmetric. You need to store data as xx, xy, yy (upper-triangular)
if enabled and as xx, xy, yx, yy (along rows then along columns, from
left to right) if disabled. Default 0, set to 1 to enable. Optional
parameter. 

uint64\_t numberKernels - specify how many kernels were initialized
before performing one input/multiple output convolutions. Overwrites
numberBatches (N). Only used in convolution step and the following
inverse transforms. Default 1. Optional parameter. 

uint64\_t kernelConvolution - specify if this application is used
to create kernel for convolution, so it has the same properties/memory
layout. performConvolution has to be set to 0 for the kernel creation.
Default 0, set to 1 to enable. Optional parameter, but it is a required
parameter for kernel generation.

\subsubsection{Register overutilization}

Only works in C2C mode, without convolution support. Enabled in Vulkan,
OpenCL and Level Zero APIs only (it works in other APIs, but worse).
Experimental feature.

uint64\_t registerBoost - specify if the register file size is bigger
than shared memory and can be used to extend it X times (on Nvidia
256KB register file can be used instead of 32KB of shared memory,
set this constant to 4 to emulate 128KB of shared memory). Default
1 - no overutilization. In Vulkan, OpenCL and Level Zero it is set
to 4 on Nvidia GPUs, to 2 if the driver shows 64KB or more of shared
memory on AMD, to 2 if the driver shows less than 64KB of shared memory
on AMD, to 1 if the driver shows 64KB or more of shared memory on
Intel, to 2 if the driver shows less than 64KB of shared memory on
Intel. Optional parameter.

uint64\_t registerBoostNonPow2 - specify if register overutilization
should be used on non-power of 2 sequences. Default 0, set to 1 to
enable. Optional parameter.

uint64\_t registerBoost4Step - specify if register file overutilization
should be used in big sequences (>2\textasciicircum 14), same definition
as registerBoost. Default 1. Optional parameter.

\subsubsection{Extra advanced parameters (filled automatically)}

uint64\_t maxComputeWorkGroupCount{[}3{]} - how many workgroups can
be launched at one dispatch. Automatically derived from the driver,
can be artificially lowered. Then VkFFT will perform a logical split
and extension of the number of workgroups to cover the required range.

uint64\_t maxComputeWorkGroupSize{[}3{]} - max dimensions of the workgroup.
Automatically derived from the driver. Can be modified if there are
some issues with the driver (as there were with ROCm 4.0, when it
returned 1024 for maxComputeWorkGroupSize and actually supported only
up to 256 threads).

uint64\_t maxThreadsNum - max number of threads per block. Similar
to maxComputeWorkGroupSize, but aggregated. Automatically derived
from the driver.

uint64\_t sharedMemorySizeStatic - available for static allocation
shared memory size, in bytes. Automatically derived from the driver.
Can be controlled by the user, if desired.

uint64\_t sharedMemorySize - available for allocation shared memory
size, in bytes. VkFFT uses dynamic shared memory in CUDA/HIP as it
allows for bigger allocations. Automatically derived from the driver.
Can be controlled by the user, if desired.

uint64\_t sharedMemorySizePow2 - the power of 2 which is less or equal
to sharedMemorySize, in bytes. Automatically computed.

uint64\_t warpSize - number of threads per warp/wavefront. Automatically
derived from the driver, but can be modified (can increase performance,
though unpredictable as defaults have good values). Must be a power
of two.

uint64\_t halfThreads - Intel GPU fix, tries to reduce the amount
of dispatched threads in half to solve performance degradation in
the Four Step FFT algorithm. Default 0 for other GPUs, try enabling
it if performance degrades in the Four Step FFT algorithm for your
GPU as well. 

int64\_t maxCodeLength - specify how big can the buffer used for code
generation be (in char). Default 1000000 chars. 

int64\_t maxTempLength - specify how big can the buffer used for intermediate
string sprintf's be (in char). Default 5000 chars. If code segfaults
for some reason - try increasing this number.

\newpage{}

\section{VkFFT Benchmark/Precision Suite and utils\_VkFFT helper routines}

The only licensed (MIT) part of the VkFFT repository is the VkFFT
header file - core library. Other files are either external helper
libraries (half, glslang, with their respective licenses) or unlicensed
code that is intended for simple copy-pasting (benchmark\_scripts,
utils\_VkFFT.h). It is the easiest way to understand how to use VkFFT
by taking the provided scripts and tinker them to the particular task.
The current version of the benchmark and precision verification suite
has the following codes available:
\begin{itemize}
\item user\_benchmark\_VkFFT - generalization of the main configuration
parameters that can be used to launch simplest in-place transforms
for the most important supported functionality
\item Sample 0 - FFT + iFFT C2C benchmark 1D batched in single precision
\item Sample 1 - FFT + iFFT C2C benchmark 1D batched in double precision
\item Sample 2 - FFT + iFFT C2C benchmark 1D batched in half precision
\item Sample 3 - FFT + iFFT C2C multidimensional benchmark in single precision
\item Sample 4 - FFT + iFFT C2C multidimensional benchmark in single precision,
native zeropadding
\item Sample 5 - FFT + iFFT C2C benchmark 1D batched in single precision,
no reshuffling
\item Sample 6 - FFT + iFFT R2C / C2R benchmark, in-place.
\item Sample 7 - FFT + iFFT C2C Bluestein benchmark in single precision
\item Sample 8 - FFT + iFFT C2C Bluestein benchmark in double precision
\item Sample 10 - multiple buffers (4 by default) split version of benchmark
0
\item Sample 11 - VkFFT / xFFT / FFTW C2C precision test in single precision
(xFFT can be cuFFT or rocFFT)
\item Sample 12 - VkFFT / xFFT / FFTW C2C precision test in double precision
(xFFT can be cuFFT or rocFFT)
\item Sample 13 - VkFFT / cuFFT / FFTW C2C precision test in half precision
\item Sample 14 - VkFFT / FFTW C2C radix 3 / 5 / 7 / 11 / 13 / Bluestein
precision test in single precision
\item Sample 15 - VkFFT / xFFT / FFTW R2C+C2R precision test in single precision,
out-of-place. (xFFT can be cuFFT or rocFFT)
\item Sample 16 - VkFFT / FFTW R2R DCT-I, II, III and IV precision test
in single precision
\item Sample 17 - VkFFT / FFTW R2R DCT-I, II, III and IV precision test
in double precision
\item Sample 18 - VkFFT / FFTW C2C radix 3 / 5 / 7 / 11 / 13 / Bluestein
precision test in double precision
\item Sample 50 - convolution example with identity kernel
\item Sample 51 - zero padding convolution example with identity kernel
\item Sample 52 - batched convolution example with identity kernel
\item Sample 100 - VkFFT FFT + iFFT R2R DCT multidimensional benchmark in
single precision
\item Sample 101 - VkFFT FFT + iFFT R2R DCT multidimensional benchmark in
double precision
\item Sample 1000 - FFT + iFFT C2C benchmark 1D batched in single precision:
all supported systems from 2 to 4096
\item Sample 1001 - FFT + iFFT C2C benchmark 1D batched in single precision:
all supported systems from 2 to 4096
\item Sample 1003 - FFT + iFFT C2C benchmark 1D batched in single precision:
all supported systems from 2 to 4096
\end{itemize}

\subsection{utils\_VkFFT helper routines}

Launching even the simplest Vulkan application can be a non-trivial
task. To help with this, utils\_VkFFT contains the routines that can
help to create the simplest Vulkan application, allocate memory, record
command buffers and launch them. Code has some comments explaining
what is going on at each step. It also has some useful struct defines
(like vkGPU) that keep the most important handles used in Vulkan Compute.
This section may be expanded in the future to the proper step-by-step
guide on Vulkan Compute simple application creation. I also encourage
to check https://github.com/DTolm/VulkanComputeSamples-Transposition
repository for another example of a compute algorithm (matrix transposition)
implemented with Vulkan API.

utils\_VkFFT also has a routine that prints the list of available
devices.

vkGPU struct has the following definition:

\begin{mdframed}[backgroundcolor=bg]
\begin{minted}[tabsize=4,obeytabs,breaklines]{C}
typedef struct {
#if(VKFFT_BACKEND==0) //Vulkan API
VkInstance instance; //a connection between the application and the Vulkan library
VkPhysicalDevice physicalDevice; //a handle for the graphics card used in the application
VkPhysicalDeviceProperties physicalDeviceProperties; //bastic device properties
VkPhysicalDeviceMemoryProperties physicalDeviceMemoryProperties; //basic memory properties of the device
VkDevice device; //a logical device, interacting with physical device
VkDebugUtilsMessengerEXT debugMessenger; //extension for debugging
uint64_t queueFamilyIndex; //if multiple queues are available, specify the used one
VkQueue queue; //a place, where all operations are submitted
VkCommandPool commandPool; //an opaque objects that command buffer memory is allocated from
VkFence fence; //a vkGPU->fence used to synchronize dispatches
std::vector<const char*> enabledDeviceExtensions;
uint64_t enableValidationLayers;
#elif(VKFFT_BACKEND==1) //CUDA API
CUdevice device;
CUcontext context;
#elif(VKFFT_BACKEND==2) //HIP API
hipDevice_t device;
hipCtx_t context;
#elif(VKFFT_BACKEND==3) //OpenCL API
cl_platform_id platform;
cl_device_id device;
cl_context context;
cl_command_queue commandQueue;
#elif(VKFFT_BACKEND==4) //Level Zero API
ze_driver_handle_t driver;
ze_device_handle_t device;
ze_context_handle_t context;
ze_command_queue_handle_t commandQueue;
uint32_t commandQueueID;
#endif
uint64_t device_id; //an id of a device, reported by devices_list call
} VkGPU;
\end{minted}
\end{mdframed}\newpage{}

\section{VkFFT Code Examples}

This section will provide some simple pseudocode for VkFFT usage,
which will once again outline important steps required to launch FFT
with VkFFT. More information (and fully working code) can be found
in this folder of the VkFFT repository:

/benchmark\_samples/vkFFT\_scripts/src/

\subsection{Driver initializations}

Before launching VkFFT, do not forget to do all necessary driver initializations.
The following code specifies them for all the supported backends,
though the final implementation may be different depending on the
particular user's configuration.

\begin{mdframed}[backgroundcolor=bg]
\begin{minted}[tabsize=4,obeytabs,breaklines]{C}
#if(VKFFT_BACKEND==0) //Vulkan API
VkResult res = VK_SUCCESS;
//create instance - a connection between the application and the Vulkan library
res = createInstance(vkGPU, sample_id);
if (res != 0) {
	//printf("Instance creation failed, error code: %" PRIu64 "\n", res);
	return VKFFT_ERROR_FAILED_TO_CREATE_INSTANCE;
} 	
//set up the debugging messenger
res = setupDebugMessenger(vkGPU);
if (res != 0) {
//printf("Debug messenger creation failed, error code: %" PRIu64 "\n", res);
	return VKFFT_ERROR_FAILED_TO_SETUP_DEBUG_MESSENGER;
}
//check if there are GPUs that support Vulkan and select one
res = findPhysicalDevice(vkGPU);
if (res != 0) {
//printf("Physical device not found, error code: %" PRIu64 "\n", res);
	return VKFFT_ERROR_FAILED_TO_FIND_PHYSICAL_DEVICE;
}
//create logical device representation
res = createDevice(vkGPU, sample_id);
if (res != 0) {
//printf("Device creation failed, error code: %" PRIu64 "\n", res);
	return VKFFT_ERROR_FAILED_TO_CREATE_DEVICE;
}
//create fence for synchronization
res = createFence(vkGPU);
if (res != 0) {
//printf("Fence creation failed, error code: %" PRIu64 "\n", res);
	return VKFFT_ERROR_FAILED_TO_CREATE_FENCE;
}
//create a place, command buffer memory is allocated from
res = createCommandPool(vkGPU);
if (res != 0) {
	//printf("Fence creation failed, error code: %" PRIu64 "\n", res);
	return VKFFT_ERROR_FAILED_TO_CREATE_COMMAND_POOL;
}
vkGetPhysicalDeviceProperties(vkGPU->physicalDevice, &vkGPU->physicalDeviceProperties);
vkGetPhysicalDeviceMemoryProperties(vkGPU->physicalDevice, &vkGPU->physicalDeviceMemoryProperties);
glslang_initialize_process();
//compiler can be initialized before VkFFT

#elif(VKFFT_BACKEND==1) //CUDA API
CUresult res = CUDA_SUCCESS;
cudaError_t res2 = cudaSuccess;
res = cuInit(0);
if (res != CUDA_SUCCESS) return VKFFT_ERROR_FAILED_TO_INITIALIZE;
res2 = cudaSetDevice((int)vkGPU->device_id);
if (res2 != cudaSuccess) return VKFFT_ERROR_FAILED_TO_SET_DEVICE_ID;
res = cuDeviceGet(&vkGPU->device, (int)vkGPU->device_id);
if (res != CUDA_SUCCESS) return VKFFT_ERROR_FAILED_TO_GET_DEVICE;
res = cuCtxCreate(&vkGPU->context, 0, (int)vkGPU->device);
if (res != CUDA_SUCCESS) return VKFFT_ERROR_FAILED_TO_CREATE_CONTEXT;
#elif(VKFFT_BACKEND==2) //HIP API
hipError_t res = hipSuccess;
res = hipInit(0);
if (res != hipSuccess) return VKFFT_ERROR_FAILED_TO_INITIALIZE;
res = hipSetDevice((int)vkGPU->device_id);
if (res != hipSuccess) return VKFFT_ERROR_FAILED_TO_SET_DEVICE_ID;
res = hipDeviceGet(&vkGPU->device, (int)vkGPU->device_id);
if (res != hipSuccess) return VKFFT_ERROR_FAILED_TO_GET_DEVICE; 
res = hipCtxCreate(&vkGPU->context, 0, (int)vkGPU->device);
if (res != hipSuccess) return VKFFT_ERROR_FAILED_TO_CREATE_CONTEXT;
#elif(VKFFT_BACKEND==3) //OpenCL API
cl_int res = CL_SUCCESS;
cl_uint numPlatforms;
res = clGetPlatformIDs(0, 0, &numPlatforms);
if (res != CL_SUCCESS) return VKFFT_ERROR_FAILED_TO_INITIALIZE;
cl_platform_id* platforms = (cl_platform_id*)malloc(sizeof(cl_platform_id) * numPlatforms);
if (!platforms) return VKFFT_ERROR_MALLOC_FAILED;
res = clGetPlatformIDs(numPlatforms, platforms, 0);
if (res != CL_SUCCESS) return VKFFT_ERROR_FAILED_TO_INITIALIZE;
uint64_t k = 0;
for (uint64_t j = 0; j < numPlatforms; j++) {
	cl_uint numDevices;
	res = clGetDeviceIDs(platforms[j], CL_DEVICE_TYPE_ALL, 0, 0, &numDevices);
	cl_device_id* deviceList = (cl_device_id*)malloc(sizeof(cl_device_id) * numDevices);
	if (!deviceList) return VKFFT_ERROR_MALLOC_FAILED;
	res = clGetDeviceIDs(platforms[j], CL_DEVICE_TYPE_ALL, numDevices, deviceList, 0);
	if (res != CL_SUCCESS) return VKFFT_ERROR_FAILED_TO_GET_DEVICE;
	for (uint64_t i = 0; i < numDevices; i++) {
		if (k == vkGPU->device_id) {
			vkGPU->platform = platforms[j];
			vkGPU->device = deviceList[i];
			vkGPU->context = clCreateContext(NULL, 1, &vkGPU->device, NULL, NULL, &res);
			if (res != CL_SUCCESS) return VKFFT_ERROR_FAILED_TO_CREATE_CONTEXT;
			cl_command_queue commandQueue = clCreateCommandQueue(vkGPU->context, vkGPU->device, 0, &res);
			if (res != CL_SUCCESS) return VKFFT_ERROR_FAILED_TO_CREATE_COMMAND_QUEUE;
			vkGPU->commandQueue = commandQueue;
			k++;
		}
		else {
			k++;
		}
	}
	free(deviceList);
}
free(platforms);
#elif(VKFFT_BACKEND==4)
	ze_result_t res = ZE_RESULT_SUCCESS;
	res = zeInit(0);
	if (res != ZE_RESULT_SUCCESS) return VKFFT_ERROR_FAILED_TO_INITIALIZE;
	uint32_t numDrivers = 0;
	res = zeDriverGet(&numDrivers, 0);
	if (res != ZE_RESULT_SUCCESS) return VKFFT_ERROR_FAILED_TO_INITIALIZE;
	ze_driver_handle_t* drivers = (ze_driver_handle_t*)malloc(numDrivers * sizeof(ze_driver_handle_t));
	if (!drivers) return VKFFT_ERROR_MALLOC_FAILED;
	res = zeDriverGet(&numDrivers, drivers);
	if (res != ZE_RESULT_SUCCESS) return VKFFT_ERROR_FAILED_TO_INITIALIZE;
	uint64_t k = 0;
	for (uint64_t j = 0; j < numDrivers; j++) {
		uint32_t numDevices = 0;
		res = zeDeviceGet(drivers[j], &numDevices, nullptr);
		if (res != ZE_RESULT_SUCCESS) return VKFFT_ERROR_FAILED_TO_GET_DEVICE;
		ze_device_handle_t* deviceList = (ze_device_handle_t*)malloc(numDevices * sizeof(ze_device_handle_t));
		if (!deviceList) return VKFFT_ERROR_MALLOC_FAILED;
		res = zeDeviceGet(drivers[j], &numDevices, deviceList);
		if (res != ZE_RESULT_SUCCESS) return VKFFT_ERROR_FAILED_TO_GET_DEVICE;
		for (uint64_t i = 0; i < numDevices; i++) {
			if (k == vkGPU->device_id) {
				vkGPU->driver = drivers[j];
				vkGPU->device = deviceList[i];
				ze_context_desc_t contextDescription = {};
				contextDescription.stype = ZE_STRUCTURE_TYPE_CONTEXT_DESC;
				res = zeContextCreate(vkGPU->driver, &contextDescription, &vkGPU->context);
				if (res != ZE_RESULT_SUCCESS) return VKFFT_ERROR_FAILED_TO_CREATE_CONTEXT;

				uint32_t queueGroupCount = 0;
				res = zeDeviceGetCommandQueueGroupProperties(vkGPU->device, &queueGroupCount, 0);
				if (res != ZE_RESULT_SUCCESS) return VKFFT_ERROR_FAILED_TO_CREATE_COMMAND_QUEUE;

				ze_command_queue_group_properties_t* cmdqueueGroupProperties = (ze_command_queue_group_properties_t*) malloc(queueGroupCount * sizeof(ze_command_queue_group_properties_t));
				if (!cmdqueueGroupProperties) return VKFFT_ERROR_MALLOC_FAILED;
				res = zeDeviceGetCommandQueueGroupProperties(vkGPU->device, &queueGroupCount, cmdqueueGroupProperties);
				if (res != ZE_RESULT_SUCCESS) return VKFFT_ERROR_FAILED_TO_CREATE_COMMAND_QUEUE;

				uint32_t commandQueueID = -1;
				for (uint32_t i = 0; i < queueGroupCount; ++i) {
					if ((cmdqueueGroupProperties[i].flags && ZE_COMMAND_QUEUE_GROUP_PROPERTY_FLAG_COMPUTE) && (cmdqueueGroupProperties[i].flags && ZE_COMMAND_QUEUE_GROUP_PROPERTY_FLAG_COPY)) {
						commandQueueID = i;
						break;
					}
				}
				if (commandQueueID == -1) return VKFFT_ERROR_FAILED_TO_CREATE_COMMAND_QUEUE;
				vkGPU->commandQueueID = commandQueueID;
				ze_command_queue_desc_t commandQueueDescription = {};
				commandQueueDescription.stype = ZE_STRUCTURE_TYPE_COMMAND_QUEUE_DESC;
				commandQueueDescription.ordinal = commandQueueID;
				commandQueueDescription.priority = ZE_COMMAND_QUEUE_PRIORITY_NORMAL;
				commandQueueDescription.mode = ZE_COMMAND_QUEUE_MODE_DEFAULT;
				res = zeCommandQueueCreate(vkGPU->context, vkGPU->device, &commandQueueDescription, &vkGPU->commandQueue);
				if (res != ZE_RESULT_SUCCESS) return VKFFT_ERROR_FAILED_TO_CREATE_COMMAND_QUEUE;
				free(cmdqueueGroupProperties);
				k++;
			}
			else {
				k++;
			}
		}

		free(deviceList);
	}
	free(drivers);
#endif
\end{minted}
\end{mdframed}

\subsection{Simple FFT application example: 1D (one dimensional) C2C (complex
to complex) FP32 (single precision) FFT}

This example performs the simplest case of FFT. It shows all the necessary
fields that the user must fill during the configuration and the submission
process. Other samples will build on this one, as driver parameters
initialization and code execution commands are the same for all configurations
(except for the launch parameters that can be configured after application
creation).

\begin{mdframed}[backgroundcolor=bg]
\begin{minted}[tabsize=4,obeytabs,breaklines]{C}
//zero-initialize configuration + FFT application
VkFFTConfiguration configuration = {};
VkFFTApplication app = {};

configuration.FFTdim = 1; //FFT dimension, 1D, 2D or 3D
configuration.size[0] = Nx; //FFT size
uint64_t bufferSize = (uint64_t)sizeof(float) * 2 * configuration.size[0];

//Device management + code submission
configuration.device = &vkGPU->device; 

#if(VKFFT_BACKEND==0) //Vulkan API		
configuration.queue = &vkGPU->queue;
configuration.fence = &vkGPU->fence; 			
configuration.commandPool = &vkGPU->commandPool; 			
configuration.physicalDevice = &vkGPU->physicalDevice; 			
configuration.isCompilerInitialized = isCompilerInitialized; //glslang compiler can be initialized before VkFFT plan creation. if not, VkFFT will create and destroy one after initialization 
#elif(VKFFT_BACKEND==3) //OpenCL API		
configuration.context = &vkGPU->context; 
#elif(VKFFT_BACKEND==4)
configuration.context = &vkGPU->context;
configuration.commandQueue = &vkGPU->commandQueue;
configuration.commandQueueID = vkGPU->commandQueueID;
#endif

allocateBuffer(buffer, bufferSize); //Pseudocode for buffer allocation, differs between APIs
transferDataFromCPU(buffer, cpu_buffer); //Pseudocode for data transfer from CPU to GPU, differs between APIs

#if(VKFFT_BACKEND==0) //Vulkan API needs bufferSize at initialization	
configuration.bufferSize = &bufferSize; 
#endif

VkFFTResult resFFT = initializeVkFFT(&app, configuration);

VkFFTLaunchParams launchParams = {};
launchParams.buffer = &buffer;
#if(VKFFT_BACKEND==0) //Vulkan API 
launchParams.commandBuffer = &commandBuffer;
#elif(VKFFT_BACKEND==3) //OpenCL API
launchParams.commandQueue = &commandQueue;
#elif(VKFFT_BACKEND==4) //Level Zero API
launchParams->commandList = &commandList;
#endif
resFFT = VkFFTAppend(app, -1, &launchParams);

//add synchronization relevant to your API - vkWaitForFences/cudaDeviceSynchronize/hipDeviceSynchronize/clFinish
transferDataToCPU(cpu_buffer, buffer); //Pseudocode for data transfer from GPU to CPU, differs between APIs

freeBuffer(buffer, bufferSize); //Pseudocode for buffer deallocation, differs between APIs

deleteVkFFT(&app);
\end{minted}
\end{mdframed} 

\subsection{Advanced FFT application example: ND, C2C/R2C/R2R, different precisions,
batched FFT}

This example shows how to configure the main parameters of interest
in the VkFFT library: multidimensional case, different types of transforms,
different precision, perform batched transforms.

In the code below X, Y and Z are the dimensions of FFT, B - number
of batches, R2C - real to complex mode 0 or 1 (on/off), DCT - 0, 1,
2, 3 or 4 (off/DCT type), P - precision (0 - single, 1 - double, 2
- half).

\begin{mdframed}[backgroundcolor=bg]
\begin{minted}[tabsize=4,obeytabs,breaklines]{C}
//zero-initialize configuration + FFT application
VkFFTConfiguration configuration = {};
VkFFTApplication app = {};

configuration.FFTdim = 1; //FFT dimension, 1D, 2D or 3D
configuration.size[0] = X;
configuration.size[1] = Y;
configuration.size[2] = Z;
if (Y > 1) configuration.FFTdim++;
if (Z > 1) configuration.FFTdim++;
configuration.numberBatches = B;
configuration.performR2C = R2C;
configuration.performDCT = DCT;
if (P == 1) configuration.doublePrecision = 1; 
if (P == 2) configuration.halfPrecision = 1;

uint64_t bufferSize = 0;

if (R2C) {
	bufferSize = (uint64_t)(storageComplexSize / 2) * (configuration.size[0] + 2) * configuration.size[1] * configuration.size[2] * configuration.numberBatches;
}
else {
	if (DCT) {
		bufferSize = (uint64_t)(storageComplexSize / 2) * configuration.size[0] * configuration.size[1] * configuration.size[2] * configuration.numberBatches;
	}
	else {
		bufferSize = (uint64_t)storageComplexSize * configuration.size[0] * configuration.size[1] * configuration.size[2] * configuration.numberBatches;
	}
} // storageComplexSize - 4/8/16 for FP16/FP32/FP64 respectively.

//Device management + code submission - code is identical to the previous example
\end{minted}
\end{mdframed}

\subsection{Advanced FFT application example: out-of-place R2C FFT with custom
strides}

In this example, VkFFT is configured to calculate a 3D out-of-place
R2C FFT of a system with custom strides. VkFFT reads data from the
inputBuffer and produces the result in the buffer. 

\begin{mdframed}[backgroundcolor=bg]
\begin{minted}[tabsize=4,obeytabs,breaklines]{C}
//zero-initialize configuration + FFT application
VkFFTConfiguration configuration = {};
VkFFTApplication app = {};

configuration.FFTdim = 3; //FFT dimension, 1D, 2D or 3D
configuration.size[0] = Nx;
configuration.size[1] = Ny;
configuration.size[2] = Nz;

configuration.performR2C = 1;

//out-of-place - we need to specify that input buffer is separate from the main buffer
configuration.isInputFormatted = 1;
configuration.inputBufferStride[0] = configuration.size[0];
configuration.inputBufferStride[1] = configuration.inputBufferStride[0] * configuration.size[1];
configuration.inputBufferStride[2] = configuration.inputBufferStride[1] * configuration.size[2];
			
configuration.bufferStride[0] = (uint64_t) (configuration.size[0] / 2) + 1;
configuration.bufferStride[1] = configuration.bufferStride[0] * configuration.size[1];
configuration.bufferStride[2] = configuration.bufferStride[1]* configuration.size[2];

uint64_t inputBufferSize = (uint64_t)sizeof(float) * configuration.size[0] * configuration.size[1] * configuration.size[2];

uint64_t bufferSize = (uint64_t)sizeof(float) * 2 * (configuration.size[0]/2+1) * configuration.size[1] * configuration.size[2];

//Device management + code submission - code is identical to the first example, except that you need to allocate two buffers (and provide them in the launch configuration).

\end{minted}
\end{mdframed}

\subsection{Advanced FFT application example: 3D zero-padded FFT}

In this example, VkFFT is configured to calculate a 3D FFT of a system.
The meaningful data is located in the first octant of the buffer,
the rest is padded with zeros. This configuration removes the circular
part of the convolution and allows modelling of open systems.

\begin{mdframed}[backgroundcolor=bg]
\begin{minted}[tabsize=4,obeytabs,breaklines]{C}
//zero-initialize configuration + FFT application
VkFFTConfiguration configuration = {};
VkFFTApplication app = {};

configuration.FFTdim = 3; //FFT dimension, 1D, 2D or 3D
configuration.size[0] = Nx;
configuration.size[1] = Ny;
configuration.size[2] = Nz;

configuration.performZeropadding[0] = 1; //Perform padding with zeros on GPU. Still need to properly align input data (no need to fill padding area with meaningful data) but this will increase performance due to the lower amount of the memory reads/writes and omitting sequences only consisting of zeros.
configuration.performZeropadding[1] = 1;
configuration.performZeropadding[2] = 1;
configuration.fft_zeropad_left[0] = (uint64_t)ceil(configuration.size[0] / 2.0);
configuration.fft_zeropad_right[0] = configuration.size[0];
configuration.fft_zeropad_left[1] = (uint64_t)ceil(configuration.size[1] / 2.0);
configuration.fft_zeropad_right[1] = configuration.size[1];
configuration.fft_zeropad_left[2] = (uint64_t)ceil(configuration.size[2] / 2.0);
configuration.fft_zeropad_right[2] = configuration.size[2];

uint64_t bufferSize = (uint64_t)storageComplexSize * configuration.size[0] * configuration.size[1] * configuration.size[2];


//Device management + code submission - code is identical to the first example

\end{minted}
\end{mdframed}

\subsection{Convolution application example: 3x3 matrix-vector convolution in
1D}

In this example, VkFFT is configured to calculate a kernel, represented
by a 3x3 matrix and a system, represented by a 3D vector. Their convolution
is a matrix-vector multiplication in the frequency domain.

\begin{mdframed}[backgroundcolor=bg]
\begin{minted}[tabsize=4,obeytabs,breaklines]{C}
//zero-initialize configuration + FFT application, we need two - one for kernel calculation
VkFFTConfiguration kernel_configuration = {};
VkFFTConfiguration convolution_configuration = {};
VkFFTApplication app_kernel = {};
VkFFTApplication app_convolution = {};


kernel_configuration.FFTdim = 1; //FFT dimension, 1D, 2D or 3D
kernel_configuration.size[0] = Nx; //FFT size

uint64_t bufferSize = (uint64_t)sizeof(float) * 2 * kernel_configuration.size[0];

//configure kernel 
kernel_configuration.kernelConvolution = 1; //specify if this plan is used to create kernel for convolution
kernel_configuration.coordinateFeatures = 9; //Specify dimensionality of the input feature vector (default 1). Each component is stored not as a vector, but as a separate system and padded on it's own according to other options (i.e. for x*y system of 3-vector, first x*y elements correspond to the first dimension, then goes x*y for the second, etc).
//coordinateFeatures number is an important constant for convolution. If we perform 1x1 convolution, it is equal to number of features, but matrixConvolution should be equal to 1. For matrix convolution, it must be equal to matrixConvolution parameter. If we perform 2x2 convolution, it is equal to 3 for symmetric kernel (stored as xx, xy, yy) and 4 for nonsymmetric (stored as xx, xy, yx, yy). Similarly, 6 (stored as xx, xy, xz, yy, yz, zz) and 9 (stored as xx, xy, xz, yx, yy, yz, zx, zy, zz) for 3x3 convolutions. 
kernel_configuration.normalize = 1;

//Initialize app_kernel and perform a single forward FFT like in examples before. You pass kernel as a buffer for the preparation stage.

convolution_configuration = kernel_configuration;
convolution_configuration.kernelConvolution = 0;
convolution_configuration.performConvolution = 1;
convolution_configuration.symmetricKernel = 0;//Specify if convolution kernel is symmetric. In this case we only pass upper triangle part of it in the form of: (xx, xy, yy) for 2d and (xx, xy, xz, yy, yz, zz) for 3d.
convolution_configuration.matrixConvolution = 3;//we do matrix convolution, so kernel is 9 numbers (3x3), but vector dimension is 3
convolution_configuration.coordinateFeatures = 3;//equal to matrixConvolution size

//Initialize app_convolution and perform a single forward FFT like in examples before. You pass kernel as kernel and system to be convolved with it as buffer

\end{minted}
\end{mdframed}

\subsection{Convolution application example: R2C cross-correlation between two
sets of N images}

In this example, VkFFT is configured to calculate a kernel, represented
by three 2D vectors (RGB values of a pixel) and a system, also represented
by three 2D vectors. There are N kernels and N systems. Their cross-correlation
is a conjugate convolution in the frequency domain. Images are usually
stored as real, not complex numbers, so code uses R2C optimization
as well.

\begin{mdframed}[backgroundcolor=bg]
\begin{minted}[tabsize=4,obeytabs,breaklines]{C}
//zero-initialize configuration + FFT application, we need two - one for kernel calculation
VkFFTConfiguration kernel_configuration = {};
VkFFTConfiguration convolution_configuration = {};
VkFFTApplication app_kernel = {};
VkFFTApplication app_convolution = {};


kernel_configuration.FFTdim = 2; //FFT dimension, 1D, 2D or 3D
kernel_configuration.size[0] = Nx;
kernel_configuration.size[1] = Ny; 
kernel_configuration.coordinateFeatures = 3;
kernel_configuration.numberBatches = N;
kernel_configuration.performR2C = 1;
kernel_configuration.normalize = 1;

uint64_t bufferSize = (uint64_t)sizeof(float) * 2 * (kernel_configuration.size[0]/2+1) * kernel_configuration.size[1] * kernel_configuration.coordinateFeatures * kernel_configuration.numberBatches; 
	
kernel_configuration.kernelConvolution = 1; //specify if this plan is used to create kernel for convolution

//Initialize app_kernel and perform a single forward FFT like in examples before. Pad in-place R2C system like this:

for (uint64_t n = 0; n < kernel_configuration.numberBatches; n++) {
	for (uint64_t c = 0; c < kernel_configuration.coordinateFeatures; c++) {
		for (uint64_t j = 0; j < kernel_configuration.size[1]; j++) {
			for (uint64_t i = 0; i < kernel_configuration.size[0]; i++) {
				kernel_padded_GPU[i + j * 2 * (kernel_configuration.size[0]/2 + 1) + c * 2 * (kernel_configuration.size[0]/2 + 1) * kernel_configuration.size[1] + n * 2 * (kernel_configuration.size[0]/2 + 1) * kernel_configuration.size[1] * kernel_configuration.coordinateFeatures] = kernel_input[i + j * kernel_configuration.size[0] + c * kernel_configuration.size[0] * kernel_configuration.size[1] + n * kernel_configuration.size[0] * kernel_configuration.size[1] * kernel_configuration.coordinateFeatures];
			}
		}
	}
}
convolution_configuration = kernel_configuration;
convolution_configuration.kernelConvolution = 0;
convolution_configuration.performConvolution = 1;
convolution_configuration.conjugateConvolution = 1;

//Initialize app_convolution and perform a single forward FFT like in examples before. Pad the system in the same way as the kernel

\end{minted}
\end{mdframed}

\subsection{Simple FFT application binary reuse application}

This example shows how to save/load binaries generated by VkFFT. This
can reduce time taken by initializeVkFFT call by removing RTC components
from it. Be sure that rest of the configuration stays the same to
reuse the binary.

\begin{mdframed}[backgroundcolor=bg]
\begin{minted}[tabsize=4,obeytabs,breaklines]{C}
VkFFTConfiguration configuration = {};
VkFFTApplication app = {};

//configuration is initialized like in other examples
configuration.saveApplicationToString = 1;
//configuration.loadApplicationFromString = 1; //choose one to save/load binary file

if (configuration.loadApplicationFromString) {
	FILE* kernelCache;
	uint64_t str_len;
#if((VKFFT_BACKEND==0) || (VKFFT_BACKEND==2) || (VKFFT_BACKEND==4))
	kernelCache = fopen("VkFFT_binary", "rb"); //Vulkan and HIP backends load data as a uint32_t sequence
#else
	kernelCache = fopen("VkFFT_binary", "r"); 
#endif
	fseek(kernelCache, 0, SEEK_END);
	str_len = ftell(kernelCache);
	fseek(kernelCache, 0, SEEK_SET);
	configuration.loadApplicationString = malloc(str_len);
	fread(configuration.loadApplicationString, str_len, 1, kernelCache);
	fclose(kernelCache);
}

resFFT = initializeVkFFT(&app, configuration);
if (resFFT != VKFFT_SUCCESS) return resFFT;

if (configuration.loadApplicationFromString)
	free(configuration.loadApplicationString);

if (configuration.saveApplicationToString) {
	FILE* kernelCache;
#if((VKFFT_BACKEND==0) || (VKFFT_BACKEND==2) || (VKFFT_BACKEND==4))
	kernelCache = fopen("VkFFT_binary", "wb"); //Vulkan and HIP backends save data as a uint32_t sequence
#else
	kernelCache = fopen("VkFFT_binary", "w"); 
#endif
	fwrite(app.saveApplicationString, app.applicationStringSize, 1, kernelCache);
	fclose(kernelCache);
}

//application is launched like in other examples
\end{minted}
\end{mdframed} 
\end{document}
